\appendix
\section{Hierarchical Bayesian Inference} \label{sec:hierarchical_inference}

% This appendix goes over in detail the hierarchical bayeisan inference framework
We use hierarchical Bayesian inference to infer the population properties of CBCs. We want to infer the number density of merging CBCs 
in the universe and how this can change with their masses, spins, etc. Often times it is useful to formulate the question in terms of the 
merger rates which is the number of mergers per $Gpc^{3}$ co-moving volume per year. For a set of hyper-parameters, $\Lambda$, $\lambda$, and local ($z=0$) 
merger rate density, $\mathcal{R}_0$, we write the overall number density of BBH mergers in the universe as: 

\begin{equation} \label{number_density}
     \frac{dN(\theta, z | \mathcal{R}_0, \Lambda, \lambda)}{d\theta dz} = \frac{dV_c}{dz}\bigg(\frac{T_\mathrm{obs}}{1+z}\bigg) \frac{d\mathcal{R}(\theta, z | \mathcal{R}_0, \Lambda, \lambda)}{d\theta} = \mathcal{R}_0 p(\theta | \Lambda) p(z | \lambda)
\end{equation}

\noindent
Where up above, we denote the co-moving volume element as $dV_c$ \citep{hogg_cosmo}, and $T_\mathrm{obs}$ as the observing time period that produced the 
catalog with the related factor of $1+z$ converting this detector-frame time to source-frame. We assume a LambdaCDM cosmology using 
the cosmological parameters from \citet{Planck2015}. We model the merger rate evolving with redshift following a power law distribution: 
$p(z|\lambda) \propto \frac{dV_c}{dz}\frac{1}{1+z}(1+z)^\lambda$. When integrating equation \ref{number_density} across all $\theta$
and out to some maximum redshift, $z_\mathrm{max}$, we get the total number of CBCs in the universe out to that redshift. We follow previous notations, \
letting $\{d_i\}$ represent the set of data from $N_\mathrm{obs}$ CBCs observed with gravitational waves. The merger rate is then described as an inhomogenous 
Poisson process and after imposing the usual log-uniform prior on the merger rate, we marginalize out merger rate and arrive at the posterior
distribution of our hyper-parameters, $\Lambda$ \citep{Mandel_2019, Vitale_2021}.

\begin{equation}
    p\left(\Lambda, \lambda | \{d_i\}\right) \frac{p(\Lambda)p(\lambda)}{\xi(\Lambda,\lambda)^{N_\mathrm{obs}}} \prod_{i=1}^{N_\mathrm{obs}} \bigg[ \frac{1}{K_i} \sum_{j=1}^{K_i} \frac{p(\theta^{i,j}|\Lambda)p(z^{i,j}|\lambda)}{\pi(\theta, z^{i,j})} \bigg]
\end{equation}

\noindent
where, we replaced the integrals over each event's likelihood with ensemble averages over $K_i$ posterior samples. Above, $j$
indexes the $K_i$ posterior samples from each event and $\pi(\theta, z)$ is the default prior used by parameter estimations that 
produced the posterior samples for each event. In the analyses of GWTC-2 the default prior used is uniform in detector frame masses, 
component spins and Euclidean volume. The corresponding prior evaluated in source frame primary mass, mass ratio, component spins 
and redshift is:

\begin{equation}
    \pi(m_1, q, a_1, a_2, \cos{\theta_1}, \cos{\theta_2}, z) \propto \frac{1}{4} m_1 (1+z)^2 D_L^2(z) \frac{dD_L}{dz}
\end{equation}

\noindent where $D_L$ is the luminosity distance. To carefully incorporate selection effects to our model we need to quantify the detection efficiency,
$\xi(\Lambda, \lambda)$, of the search pipelines that were used to create GWTC-3, at a given population distribution described by $\Lambda$ and $\lambda$.
 
\begin{equation}
     \xi(\Lambda, \lambda) = \int d\theta dz P_\mathrm{det}(\theta, z)p(\theta | \Lambda) p(z | \lambda)
\end{equation}
 
\noindent
To estimate this integral we use a software injection campaign where gravitational waveforms from a large population of simulated sources. 
These simulated waevforms are put into real detector data, and then this data is evaluated with the same search pipelines that were used to 
produce the catalog we are analyzing. With these search results in hand, we use importance sampling and evaluate the integral 
with the monte carlo sum estimate $\mu$, and its corresponding variance and effective number of samples:

\begin{equation} \label{xi}
     \xi(\Lambda, \lambda) \approx \mu(\Lambda, \lambda) \frac{1}{N_\mathrm{inj}} \sum_{i=1}^{N_\mathrm{found}} \frac{p(\theta^i | \Lambda) p(z^i | \lambda)}{p_\mathrm{inj}(\theta, z^i)}
\end{equation}

\begin{equation}
    \sigma^2(\Lambda, \lambda) \equiv \frac{\mu^2(\Lambda, \lambda)}{N_\mathrm{eff}} \simeq \frac{1}{N^2_\mathrm{inj}} \sum_{i=1}^{N_\mathrm{found}} \bigg[\frac{p(\theta | \Lambda) p(z | \lambda)}{p_\mathrm{inj}(\theta, z)}\bigg]^2 - \frac{\mu^2(\Lambda, \lambda)}{N_\mathrm{inj}}
\end{equation}

\noindent
Where the sums indexes only over the $N_\mathrm{found}$ injections that were successfully detected out of $N_\mathrm{inj}$ total injections, 
and $p_\mathrm{inj}(\theta, z)$ is the reference distribution from which the injections were drawn. Additionally, we follow the procedure 
outlined in \citet{Farr_2019} to marginalize the uncertainty in our estimate of $\xi(\Lambda, \lambda)$, in which we verify that $N_\mathrm{eff}$ is 
sufficiently high after re-weighting the injections to a given population (i.e. $N_\mathrm{eff} > 4N_\mathrm{obs}$). 
The total hyper-posterior marginalized over the merger rate and the uncertainty in the monte carlo integral calculating $\xi(\Lambda, \lambda)$ \citep{Farr_2019}, as:

\begin{equation}\label{importance-posterior}
    \log p\left(\Lambda, \lambda | \{d_i\}\right) \propto \sum_{i=1}^{N_\mathrm{obs}} \log \bigg[ \frac{1}{K_i} \sum_{j=1}^{K_i} \frac{p(\theta^{i,j}|\Lambda)p(z^{i,j}|\lambda)}{\pi(\theta^{i,j}, z^{i,j})} \bigg] -  \\
    N_\mathrm{obs} \log \mu(\Lambda, \lambda) + \frac{3N_\mathrm{obs} + N_\mathrm{obs}^2}{2N_\mathrm{eff}} + \mathcal{O}(N_\mathrm{eff}^{-2}).
\end{equation}

We explicitly enumerate each of the models used in this work for $p(\theta|\Lambda)$, along with 
their respective hyper-parameters and prior distributions in the next section. To calculate draw 
samples of the hyper parameters from the hierarchical posterior distribution shown in equation \ref{importance-posterior}, we use the 
NUTS hamiltonian monte carlo sampler in \textsc{numpyro} and \textsc{jax} to calculate likelihoods \bruce{FIND citations for these}.

\section{Penalized Splines and Smoothing}\label{sec:psplines}

A common issue when using splines as a flexible fitting method is their sensitivity to the chosen number of knots, and their locations. 
Adding more knots knots will increase the natural a-priori variance in the spline function, while the space between knots can limit the 
resolution of features in the data the spline is capable of fitting. To ensure your spline based model is flexible enough to find sharp
features in data, one would want to add as many knots as densely as possible, but this comes with the sometimes un-wanted affect of the 
larger variance as mentioned above. To overcome this we resort to the frequentist penalized spline (PSpline) in which one applies a penalty
to the likelihood based on the difference of adjacent knot coeficients. Since the resulting linear combination of spline basis components is 
smoother, or flatter, when the basis coeficients are near each other, this has a natural smoothing affect to the spline function. Since we are 
using hierarchical Bayesian inferece as our statistical framework, we formulate this penalized likelihood with its Bayesian analog, by applying 
the stochastic differnece prior described in \bruce{CITE THIS PAPER}, which is equivalent to a Gaussian random walk prior where each succeeding 
coeficient is normally distributed about its neighboring coeficients. Thus the smoothing prior for $n$ degree's of freedom basis with a smoothing 
difference order of $r$ is defined as:

\begin{eqnarray}
\tau_\lambda \sim \mathcal{G}(a, b), \\
p(\bm{c} | \tau_\lambda) \propto \exp \big[ -\frac{1}{2} \tau_\lambda \bm{c}^{\mathrm{T}} \bm{D_r}^{\mathrm{T}} \bm{D_r} \bm{c}  \big] e^{-n\bm{c}}
\end{eqnarray}

\noindent Above $\bm{D_r}$ is the order-$r$ difference matrix, of shape $(n-r \times n)$ and $a$, $b$, shape parameters for the Gamma distribution, such that
the mean is at 1, with a large vairabce. This smoothing prior removes the strong dependence on number and location of knots that arises with using splines. 
By adding in a very large number of knots such that your domain is densly populated with basis coeficients, the smoothing prior will smooth out regions where 
the data can be described smoothly, and relax when the data has features for the spline function to fit. In practice one would want to add knots 
of order the number of observations or more. One downside of this prior specification is a moderate dependence on choice of the b hyper-parameter for 
Gamma distribution prior on $\tau_\lambda$. We reduce this affect by following the method in \bruce{CITE BAYESIAN PSPLINES} where we construct 
a mixture model on a wide range of gamma distributions with different b hyper-parameters. We can now replace the prior on $\tau_\lambda$ with:

\begin{align}
    \bm{p} \sim \mathcal{D}(m), \\
    p(\tau_\lambda | \bm{p}) = \sum_i^m p_i \mathcal{G}(1, b_m)
\end{align}

Additionally one can implement a spatially adaptive smoothing prior which allows for a smoothing prior
that can smoothly evolve over the domain of your spline function. This is done in practice by having the $\tau_\lambda$ parameter evolve smoothly
across the knots. \bruce{CITE BAYESIAN PSPLINES}, formulates a spatially adaptive prior in this way by adding in $n-r-1$ additional parameters that
effectively modulate the $\tau_\lambda$ across each difference. We now replace our prior on the coeficients with:

\begin{eqnarray}
\bm{\lambda} \sim \mathcal{G}(\omega, \omega), \\
\bm{\Lambda} \equiv \mathrm{diag}\big(1, \lambda_0,...,\lambda_{n-r-1}\big), \\
p(\bm{c} | \tau_\lambda) \propto \exp \big[ -\frac{1}{2} \tau_\lambda \bm{c}^{\mathrm{T}} \bm{D_r}^{\mathrm{T}} \bm{\Lambda} \bm{D_r} \bm{c} \big] e^{-n\bm{c}}
\end{eqnarray}

Where above the matrix $\bm{\Lambda}$ a diagonal matrix of shape $(n-r \times n-r)$, and $\omega$ an arbitrarily small value such that the Gamma distribution 
prior on $\bm{\lambda}$ has a mean of 1 and an arbitrarily large variance. This enforces a smooth evolution of the smoothing penalty 
across the domain that is being analyzed. This type of prior is especiially important when modeling distributions that span large orders of magnitude like 
the power-law like mass distributions that are commonly used. 

Since the smoothing prior is really wanting to pull the coeficients closer to each other to smooth the distribution, 
we must ensure that the spline function is in fact flat or smooth with all equal coeficients. We follow the distinction from \bruce{CITE bayesian psplines} that the difference
penalty or prior is only valid with ``proper'' spline bases. This is where the knots are evenly spaced and are not stacked at the extremal values, 
as is sometimes done. Figure shows examples of two different M-Spline bases with 10 degrees of freedom and knots spaced linearly from 0 to 1. The ``proper''
basis is shown on the left where the resulting (black) curve is flat with equal components, while the ``improper'' basis on the right has the 
components squeezed up against the boundaries where multiple knots are located causing the sharp increases at the bounds even with all 10 coeficients equal.
*\

\section{Model specification} \label{sec:modelpriors}

\section{Validation Studies} \label{sec:validation}
