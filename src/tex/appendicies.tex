\appendix
\section{Penalized Splines and Smoothing}\label{sec:psplines}

Spline functions have been shown to be sensitive to the chosen number of knots, and their locations or spacing \citep{deBoor78}. 
Adding more knots increases the a priori variance in the spline function, while the space between knots can limit the 
resolution of features in the data the spline is capable of resolving. To ensure your spline based model is flexible enough 
one would want to add as many knots as densely as possible, but this comes with unwanted side effect of larger variance as mentioned above. 
To overcome this we resort to the frequentest penalized spline (P-Spline) in which one applies a penalty or regularization term 
to the likelihood based on the difference of adjacent knot coefficients \citep{eilers2021practical}. The linear combination of spline basis components 
or the resulting spline function is flat when the basis coefficients are equal (see Figure \ref{fig:spline_basis}). By penalizing the likelihood as 
the differences between adjacent knot coefficients get larger, one gets a smoothing of the spline function inferences \citep{eilers2021practical}. 
Since we are using hierarchical Bayesian inference as our statistical framework, we formulate this penalized likelihood with its Bayesian analog. 
The difference penalty is equivalent to applying a stochastic difference prior among the knots, as described in \citet{BayesianPSplines} and \citet{Jullion2007RobustSO}. 
This is also sometimes referred to as a Gaussian random walk prior, since each coefficient is normally distributed about its neighboring coefficients. 
For a spline basis with $n$ degree's of freedom, and a difference penalty of order of $r$ (see \citet{eilers2021practical}), the smoothing prior is defined as:

\begin{eqnarray}
\tau_\lambda \sim \Gamma(a, b) \\
p(\bm{c} | \tau_\lambda) \propto \exp \big[ -\frac{1}{2} \tau_\lambda \bm{c}^{\mathrm{T}} \bm{D}_{r}^{\mathrm{T}} \bm{D}_r \bm{c}  \big] e^{-n\bm{c}}
\end{eqnarray}

\noindent Above $\bm{D}_r$ is the order-$r$ difference matrix, of shape $(n-r \times n)$ and $\Gamma(a,b)$ the Gamma distribution with shape parameters, $a$ and $b$. 
This smoothing prior removes the strong dependence on number and location of knots that arises with using splines. The $\tau_\lambda$ controls the ``strength'' of 
smoothing, or the inverse variance of the Gaussian priors on knot differences. By adding in a very large number 
of knots such that your domain is densely populated with basis coefficients, this allows the freedom for the model to find the smoothing 
scale that the data prefers. One downside of this prior specification as noted in \citet{BayesianPSplines} and \citet{Jullion2007RobustSO}, is a slight dependence on choice 
of the b hyperparameter for Gamma distribution prior on $\tau_\lambda$. We reduce this affect by following the method introduced in \citet{Jullion2007RobustSO}
to alleviate this. This is done by modeling $\tau_\lambda$ with a mixture model on gamma distributions with varying b hyperparameters. 
We now replace the prior on $\tau_\lambda$ with:

\begin{align}
    \bm{p} \sim \mathcal{D}(m) \\
    p(\tau_\lambda | \bm{p}) = \sum_i^m p_i \Gamma(1, b_i)
\end{align}

\noindent Where above $\mathcal{D}(m)$ is a Dirichlet distribution with $m$ degrees of freedom and ones as the concentration parameters, 
and the $b_i$'s are $m$ different b shape parameters for the Gamma distribution.

%Additionally one can implement a spatially adaptive smoothing prior which allows for a smoothing prior
%that can smoothly evolve over the domain of your spline function. This is done in practice by having the $\tau_\lambda$ parameter evolve smoothly
%across the knots. \bruce{CITE BAYESIAN PSPLINES}, formulates a spatially adaptive prior in this way by adding in $n-r-1$ additional parameters that
%effectively modulate the $\tau_\lambda$ across each difference. We now replace our prior on the coeficients with:

%\begin{eqnarray}
%\bm{\lambda} \sim \mathcal{G}(\omega, \omega), \\
%\bm{\Lambda} \equiv \mathrm{diag}\big(1, \lambda_0,...,\lambda_{n-r-1}\big), \\
%p(\bm{c} | \tau_\lambda) \propto \exp \big[ -\frac{1}{2} \tau_\lambda \bm{c}^{\mathrm{T}} \bm{D_r}^{\mathrm{T}} \bm{\Lambda} \bm{D_r} \bm{c} \big] e^{-n\bm{c}}
%\end{eqnarray}

%Where above the matrix $\bm{\Lambda}$ a diagonal matrix of shape $(n-r \times n-r)$, and $\omega$ an arbitrarily small value such that the Gamma distribution 
%prior on $\bm{\lambda}$ has a mean of 1 and an arbitrarily large variance. This enforces a smooth evolution of the smoothing penalty 
%across the domain that is being analyzed. This type of prior is especiially important when modeling distributions that span large orders of magnitude like 
%the power-law like mass distributions that are commonly used. 

Since the prior is imparting a natural attraction of the coefficients closer to each other in order to smooth the spline function, so one 
must ensure that the spline function is in fact flat given all equal coefficients. Since one needs to define n+k knots to construct an order-k basis with 
n degrees of freedom, sometimes studies place knots on top of each other at hard parameter boundaries \cite{deBoor78,monotone_regression_splines}. 
This may seem motivated, especially when is no reality in the parameter beyond the boundary, but this violates the above condition. 
We follow the distinction from \citet{eilers2021practical} that such a smoothing prior is only valid with ``proper'' spline bases. 
A proper basis is where all n+k knots are evenly spaced, see Figure \ref{fig:spline_basis}, as opposed to stacking them at the bounds.

\section{Hierarchical Bayesian Inference} \label{sec:hierarchical_inference}

% This appendix goes over in detail the hierarchical bayeisan inference framework
We use hierarchical Bayesian inference to infer the population properties of CBCs. We want to infer the number density of merging CBCs 
in the universe and how this can change with their masses, spins, etc. Often times it is useful to formulate the question in terms of the 
merger rates which is the number of mergers per $Gpc^{3}$ co-moving volume per year. For a set of hyperparameters, $\Lambda$, $\lambda$, and local ($z=0$) 
merger rate density, $\mathcal{R}_0$, we write the overall number density of BBH mergers in the universe as: 

\begin{equation} \label{number_density}
     \frac{dN(\theta, z | \mathcal{R}_0, \Lambda, \lambda)}{d\theta dz} = \frac{dV_c}{dz}\bigg(\frac{T_\mathrm{obs}}{1+z}\bigg) \frac{d\mathcal{R}(\theta, z | \mathcal{R}_0, \Lambda, \lambda)}{d\theta} = \mathcal{R}_0 p(\theta | \Lambda) p(z | \lambda)
\end{equation}

\noindent
Where up above, we denote the co-moving volume element as $dV_c$ \citep{hogg_cosmo}, and $T_\mathrm{obs}$ as the observing time period that produced the 
catalog with the related factor of $1+z$ converting this detector-frame time to source-frame. We assume a Lambda CDM cosmology using 
the cosmological parameters from \citet{Planck2015}. We model the merger rate evolving with redshift following a power law distribution: 
$p(z|\lambda) \propto \frac{dV_c}{dz}\frac{1}{1+z}(1+z)^\lambda$ \cite{Fishbach_2018redshift}. When integrating equation \ref{number_density} across all $\theta$
and out to some maximum redshift, $z_\mathrm{max}$, we get the total number of CBCs in the universe out to that redshift. We follow previous notations, \
letting $\{d_i\}$ represent the set of data from $N_\mathrm{obs}$ CBCs observed with gravitational waves. The merger rate is then described as an inhomogeneous 
Poisson process and after imposing the usual log-uniform prior on the merger rate, we marginalize over the local merger rate, $\mathcal{R}_0$, and arrive at the posterior
distribution of our hyperparameters, $\Lambda$ \citep{Mandel_2019, Vitale_2021}.

\begin{equation}
    p\left(\Lambda, \lambda | \{d_i\}\right) \frac{p(\Lambda)p(\lambda)}{\xi(\Lambda,\lambda)^{N_\mathrm{obs}}} \prod_{i=1}^{N_\mathrm{obs}} \bigg[ \frac{1}{K_i} \sum_{j=1}^{K_i} \frac{p(\theta^{i,j}|\Lambda)p(z^{i,j}|\lambda)}{\pi(\theta, z^{i,j})} \bigg]
\end{equation}

\noindent
Where above, we replaced the integrals over each event's likelihood with ensemble averages over $K_i$ posterior samples. Above, $j$
indexes the $K_i$ posterior samples from each event and $\pi(\theta, z)$ is the default prior used by parameter estimations that 
produced the posterior samples for each event. In the analyses of GWTC-3, either the default prior used was uniform in detector frame masses, 
component spins and Euclidean volume or the posterior samples were re-weighted to such a prior before using them in our analysis. 
The corresponding prior evaluated in the parameters we hierarchically model, i.e. source frame primary mass, mass ratio, component spins and redshift is:

\begin{equation}
    \pi(m_1, q, a_1, a_2, \cos{\theta_1}, \cos{\theta_2}, z) \propto \frac{1}{4} m_1 (1+z)^2 D_L^2(z) \frac{dD_L}{dz}
\end{equation}

\noindent Above, $D_L$ is the luminosity distance. To carefully incorporate selection effects to our model we need to quantify the detection efficiency,
$\xi(\Lambda, \lambda)$, of the search pipelines that were used to create GWTC-3, at a given population distribution described by $\Lambda$ and $\lambda$.
 
\begin{equation}
     \xi(\Lambda, \lambda) = \int d\theta dz P_\mathrm{det}(\theta, z)p(\theta | \Lambda) p(z | \lambda)
\end{equation}
 
\noindent
To estimate this integral we use a software injection campaign where gravitational waveforms from a large population of simulated sources. 
These simulated waveforms are put into real detector data, and then this data is evaluated with the same search pipelines that were used to 
produce the catalog we are analyzing. With these search results in hand, we use importance sampling and evaluate the integral 
with the Monte Carlo sum estimate $\mu$, and its corresponding variance and effective number of samples:

\begin{equation} \label{xi}
     \xi(\Lambda, \lambda) \approx \mu(\Lambda, \lambda) \frac{1}{N_\mathrm{inj}} \sum_{i=1}^{N_\mathrm{found}} \frac{p(\theta^i | \Lambda) p(z^i | \lambda)}{p_\mathrm{inj}(\theta, z^i)}
\end{equation}

\begin{equation}
    \sigma^2(\Lambda, \lambda) \equiv \frac{\mu^2(\Lambda, \lambda)}{N_\mathrm{eff}} \simeq \frac{1}{N^2_\mathrm{inj}} \sum_{i=1}^{N_\mathrm{found}} \bigg[\frac{p(\theta | \Lambda) p(z | \lambda)}{p_\mathrm{inj}(\theta, z)}\bigg]^2 - \frac{\mu^2(\Lambda, \lambda)}{N_\mathrm{inj}}
\end{equation}

\noindent
Where the sum is only over the $N_\mathrm{found}$ injections that were successfully detected out of $N_\mathrm{inj}$ total injections, 
and $p_\mathrm{inj}(\theta, z)$ is the reference distribution from which the injections were drawn. We use the LVK released injection sets that describe the 
detector sensitivities over the first, second and third observing runs \citep{o1o2o3_injection_sets}. Additionally, we follow the procedure 
outlined in \citet{Farr_2019} to marginalize the uncertainty in our estimate of $\xi(\Lambda, \lambda)$, in which we verify that $N_\mathrm{eff}$ is 
sufficiently high after re-weighting the injections to a given population (i.e. $N_\mathrm{eff} > 4N_\mathrm{obs}$). 
The total hyper-posterior marginalized over the merger rate and the uncertainty in the Monte Carlo integral calculating $\xi(\Lambda, \lambda)$ \citep{Farr_2019}, as:

\begin{equation}\label{importance-posterior}
    \log p\left(\Lambda, \lambda | \{d_i\}\right) \propto \sum_{i=1}^{N_\mathrm{obs}} \log \bigg[ \frac{1}{K_i} \sum_{j=1}^{K_i} \frac{p(\theta^{i,j}|\Lambda)p(z^{i,j}|\lambda)}{\pi(\theta^{i,j}, z^{i,j})} \bigg] -  \\
    N_\mathrm{obs} \log \mu(\Lambda, \lambda) + \frac{3N_\mathrm{obs} + N_\mathrm{obs}^2}{2N_\mathrm{eff}} + \mathcal{O}(N_\mathrm{eff}^{-2}).
\end{equation}

We explicitly enumerate each of the models used in this work for $p(\theta|\Lambda)$, along with 
their respective hyperparameters and prior distributions in the next section. To calculate draw 
samples of the hyperparameters from the hierarchical posterior distribution shown in equation \ref{importance-posterior}, we use the 
NUTS Hamiltonian Monte Carlo sampler in \textsc{NumPyro} and \textsc{Jax} to calculate likelihoods \citep{jax,pyro,numpyro}.

\section{Model specification} \label{sec:modelpriors}

\bruce{Need to fill this section in and put tables with specific hyperparameter/prior settings made}

%\section{Validation Studies} \label{sec:validation}

%bruce{Need to fill this section}
