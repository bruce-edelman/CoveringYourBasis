\section{Astrophysical Implications}\label{sec:astrodiscussion}

The collective distibution of BBH source properties provides a useful probe of the complex and uncertain astrophysics that governs their 
formation and evolution to eventually merge \citep{Zevin_2017}. Our analysis with the newly constructed \textsc{MSpline} models corroborates important 
conclusions of recent work and provides a robust data-driven framework for future population studies. 

% mass dist
The results presented in section \ref{sec:mass_dist} illustrate a much wider mass distribution than inferred with power-law based models in \citet{o3b_astro_dist}. 
While isolated formation is able to predict the $10\msun$ peak \bruce{cite}, cluster and dynamical formation scenarios struggle to predict a peak in the BH mass distribution less than 
$15-20\msun$ \bruce{cite}. Globular cluster formation is expected to produce more top-heavy mass distributions than isolated and recent studies have shown suppressed BBH merger rates 
at lower ($m\leq15\msun$) masses when compared to predictions from the isolated channel \bruce{cite}. BBHs that form near active galactive nuclei (AGN) can preferentially produce higher 
mass black holes \bruce{cite}. We do not find any evidence for a truncation or rapid decline in the merger rate as a function of mass, that stellar evolution theory predicts 
due to pair-instability supernovae (PISNe) \citep{Heger_2002,PISN_Woosley,Heger_2003,Spera_2017}. The original motivation for the peak in the \textsc{PLPeak} model \citep{Talbot_2018} was to represent a possible ``pileup'' of 
masses just before such the truncation, since massive stars just light enough to avoid PISN will shed large amounts of mass in a series of ``pulses'' before collapsing to BHs in a process called 
pulsational pair-instability supernova (PPISN) \citep{Woosley_2017,Woosley_2019,Farmer_2019}. While the predictions of the mass scale where pair-instability kicks in are uncertain and depend on not well understood physics, 
like nuclear reaction rates of carbon and oxygen in the core of stars, models have a hard time predicting this peak to be lower than $m\sim40\msun$ \citep{Belczynski_2016,Marchant_2019,Renzo_2020,Farmer_2019,Farmer_2020}. The lack of a truncation could 
point towards a higher prevalence of dynamical processes that can produce black holes in mass ranges stellar collapse cannot, such as hierarchical mergers of BHs \citep{Fishbach_2017,Doctor_2020,Kimball_genealogy,kimball2020evidence,doctor2021black,Fishbach_2022}, 
very low metallictiy population III stars \citep{Belczynski_2020,Farrell_2020}, new beyond the standard model physics\cite{Croon_newphysics,Sakstein_2020}, or black hole accretion of BHs in gaseous environments such as AGNs \citep{Secunda_2020,McKernan_2020,cruzosorio2021gw190521}. 

% spin dist
Section \ref{sec:spin_dist} focused on inferences of the spin distributions of black holes, observing evidence of spin mis-alignment, spin anti-alignment, and suppressed support 
for exactly aligned systems. These point towards a significant contribution of the population being formed through dynamical formation processes, agreeing with 
conclusions drawn about the mass distribution inference of section \ref{sec:mass_dist}. While field formation is expected to produce systems with preferentially 
aligned spins due to tidal interactions, but observational evidence suggests that tides may not be able to re-align spins in all systems as some 
isolated population models assume. Additionally because of uncertain knowledge of supernovae kicks, isolated formation can produce systems with negative but small effective spins. 
We report an effective spin distribution that is not symmetric about zero, disfavoring a scenarior in wihch all BBHs are formed dynamically. Following the rules in \citet{Fishbach_2022}, 
we place conservative upper bounds on the fraction of hierarhical mergers $f_\mathrm{HM}$ and fraction of dynamically formed BBHs, $f_\mathrm{dyn}$ with 
the \textsc{MSpline} $\chi_\mathrm{eff}$ model constraining $f_\mathrm{HM} < $\result{$\macros[ChiEffective][chieff][frac_hm][10th percentile]$} 
and $f_\mathrm{dyn} < $\result{$\macros[ChiEffective][chieff][frac_dyn][10th percentile]$} at 90\% credibility. 


\section{Conclusions}\label{sec:conclusion}

\bruce{main points for conclusion are to sum up conclusions with one last call to literature for most important parts, 
followed by short discussion of the outlook going into O4 and how this method will be especially useful as we rapidly approach
GW's data rich era}