\section{Introduction} \label{sec:intro}

Observations of gravitational waves (GWs) from compact binary mergers are becoming a regular occurrence, 
producing a catalog of events that recently surpassed 90 such detections \citep{GWTC1,gwtc2,GWTC3}. As the catalog continues to grow, so does our understanding of the underlying astrophysical population of compact binaries \citep{o1o2_pop,o3a_pop,o3b_astro_dist}. 
Following numerous improvements to the detectors since the last observing run, the anticipated sensitivities for the upcoming fourth observing run of the LIGO-Virgo-KAGRA (LVK) collaboration suggest detection rates as high as once per \emph{day} \citep{aLIGO,aVIRGO,LVK_prospects}. With the formation history of these dense objects encoded in the details of their distribution \citep{Zevin_2017}, the likely doubling in size of the catalog with the next observing run could provide another leap in our understanding of compact binary astrophysics. 
Beyond formation physics, population-level inference of the compact binary catalog has also been shown to provide novel measurements of cosmological parameters \citep{Farr_2019HUB,gwtc3_cosmo}, constrain modified gravitational wave propagation \citep{ModGWProp,ModGWProp2}, 
constrain a running Planck mass \citep{Lagos_runningPlanckMass}, search for evidence of ultralight bosons through superradiance \citep{Ng_Boson2021,GWTC2_superradiance_Ng}, 
constrain stellar nuclear reaction rates \citep{Farmer_2019,Farmer_2020}, look for primordial black holes \citep{Ng_2021,KenNgPBH2022}, 
and to constrain physics of neutron stars \citep{Golomb_EOS,LandryRead_NS_Masses2021}. Through a better understanding of the mass, spin, and redshift distributions of 
compact binaries that will come with the increased catalog size, one can probe a wide range of different physical phenomena with even greater fidelity.

The binary black hole (BBH) mass distribution was first found to have structure beyond a smooth power law with simpler parametric models, exhibiting a possible high mass truncation and either a break or a peak at $m_1\sim35-40\,\msun$ \citep{Fishbach_2017,Talbot_2018,o1o2_pop,o3a_pop}. Starting with moderately sized catalog, GWTC-2, more flexible models found signs of additional structure \citep{Tiwari_2021_b,Edelman_2022ApJ}. The evidence supporting these features, 
such as the peak at $m_1\sim10\,\msun$, has only grown after analyzing the latest catalog, GWTC-3, with the same models \citep{o3b_astro_dist,Tiwari_2022ApJ}. 
While this shows the usefulness of data-driven methods with the current relatively small catalog size, they will become more powerful with more observations. The canonical approach to constructing population models has been 
to use simple parametric descriptions (e.g., power laws, beta distributions) that aim to describe the data in the simplest way, employ astrophysically motivated priors where appropriate, then sequentially add 
complexities (e.g., Gaussian peaks) as the data demands. This simple approach was necessary when data was scarce, but as we move into the data-rich catalog era, this approach is already failing to scale.  More flexible and scalable methods, such as non-parametric modeling techniques, will be necessary to continue to extract the full information contained in the compact binary catalog. In contrast to parametric models, flexible and non-parametric models are data-driven and 
contribute little bias to functional form. Non-parametric or semi-parametric methods are particularly useful to search for unexpected features in the 
data, providing meaningful insight of these features that parametric models may fail to capture.

While we eventually hope to uncover hints of binary formation mechanisms in the mass spectrum of BBHs, the distribution of spin properties have been of particular interest.  
The measurement of spin properties of individual binaries often have large uncertainties, but the theorized formation channels are expected to produce distinctly 
different spin distributions \citep{Farr_BinnedSpin, Farr2017Nature, Zevin_2017}. Isolated (or field) formation scenarios predict component spins that are preferentially aligned with the binary's orbital angular momentum, although some small 
misalignment can occur depending on the nature of the supernova kicks as each star collapses to a compact object \citep{Zevin_2022, BaveraBBHSpin, BaveraMassTransfer}. Alternatively, dynamical formation in 
dense environments where many-body interactions between compact objects can result in binary formation and hardening (shrinking of binary orbits) should produce binaries with components' spins distributed isotropically \citep{Rodriguez_2016, Rodriguez_2019}. BBH spins have also been of controversial interest recently, with different parametric approaches to modeling 
the spin distribution coming to different conclusions. Studies have disagreed on the possible existence of a significant zero-spinning subpopulation, as well as the presence of 
significant spin misalignment (i.e. $\cos{\theta_i} < 0.0$) \citep{o3b_astro_dist,RouletGWTC2Pop,BuildBetterSpinModels,GWTC3MonashSpin,Callister_NoEvidence}. 
Another study recently showed that inferences of spin misalignment (or tilts) are sensitive to modeling choices and may not peak at perfectly aligned spins, as is often assumed \citep{spinitasyoulike}. While enlightening,
these recent efforts to improve BBH spin models continue to build sequentially on previous parametric descriptions \citep{BuildBetterSpinModels,Callister_NoEvidence,spinitasyoulike}. 
To ensure we are extracting the full detail the catalog has to offer, we extend our previous non-parametric modeling techniques to include spin magnitudes and tilts, as well as the binary mass ratio and redshift. 
%We present in this manuscript, both to help alleviate the current disputes in the literature and because there has been little work on flexible model development for the 
%spin distribution of BBHs beyond the aligned spin dimension thus far \citep{Tiwari_2021_a,Tiwari_2021_b}.

Polynomial splines have been applied to success across different areas of gravitational-wave astronomy. They have been used to model the gravitational-wave data noise spectrum, 
detector calibration uncertainties, coherent gravitational waveform deviations, and modulations to a power law mass distribution \citep{Littenberg_2015,Edwards_2018,B_Farr_etal_2014,Edelman_2021,Edelman_2022ApJ}
In this paper we highlight how the use of basis-splines can provide a powerful non-parametric modeling approach to the astrophysical distributions of compact 
binaries. We illustrate how one can efficiently model both the mass and spin distributions of merging compact binaries in GWTC-3 with basis splines to infer compact binary population properties using 
hierarchical Bayesian inference. We discuss our results in the context of current literature on compact object populations and how this method complements the simpler lower 
dimensional parametric models in the short run, and will become necessary with future catalogs. Should they appear with more observations, this data-driven approach will provide checks of 
our understanding by uncovering more subtle -- potentially unexpected -- features. The rest of this manuscript is structured as follows: a description of the background of 
basis splines in section \ref{sec:basis_splines}, followed by a presentation of the results of our extensive, data-driven study of the mass and spin distributions of BBHs in GWTC-3 in section 
\ref{sec:results}. We then discuss these results and their astrophysical implications in section \ref{sec:astrodiscussion} and finish with our conclusions in section \ref{sec:conclusion}.