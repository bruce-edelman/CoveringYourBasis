\section{Introduction} \label{sec:intro}

Observations of gravitational waves (GWs) from compact binary coalescences (CBCs) is becoming a regular occurrence, 
recently producing catalogs of 90 such detection \cite{GWTC1,gwtc2,GWTC3}. With this growing catalog size, we are steadily approaching a more complete  
understanding of the population properties CBCs, just 7 years after the first observation \cite{o1o2_pop,o3a_pop,o3b_astro_dist}. 
In the coming fourth observing run of the LIGO-Virgo-KAGRA collaboration detectors \citep{aLIGO,aVIRGO}, with the increased sensitivities, 
there could be a detection as often as every day \cite{LVK_prospects}. This large influx of incoming data will drastically increase what we can learn by combining all observations
in order to studying the compact object population. The history of these dense objects is imprinted on the properties of the population 
of CBCs, allowing us to learn about their formation and evolution by the last few seconds of gravitation radiation emitted. The population distibutions 
of merging compact objects inferred from gravitational-wave observations has been shown to be parituclar useful in order to, measure cosmological parameters \citep{Farr_2019HUB,gwtc3_cosmo}, 
constrain modified gravitational wave propagation \citep{ModGWProp,ModGWProp2}, constrain a running Planck mass \citep{Lagos_runningPlanckMass}, look for evidence of ultralight bosons through superradiance \citep{Ng_Boson2021,Ng_BosonGWTC2_2021}, 
constrain stellar nuclear reaction rates \citep{Farmer_2019,Farmer_2020}, look for primordial black holes \citep{Ng_2021,KenNgPBH2022}, and constrain the neutron star equation of state \citep{Golomb_EOS}. 
Through a better understanding of the mass, spin, and redshift distributions of compact binaries that will come with the increased catalog size, 
one can probe a wide range of different physical phenoma with even greater fidelity.

Studies using flexible methods to analyze the binary black hole (BBH) mass distribution with GWTC-2 \citep{Tiwari_2021_b,Edelman_2022ApJ} found signs of structure 
beyond the reported $m_1\sim35\,\msun$ feature by \citet{Talbot_2019, o3a_pop}. The has evidence for this structure, such as the peak at $m_1\sim10\,\msun$, has only increased
with GWTC-3 \cite{o3b_astro_dist,Tiwari_2022ApJ}. This substructure was first confidently found with mutliple different flexible or non-parametric models that aim 
to be data-oriented with little bias on the functional form. Conversely the other canonical approach is to use simple parametric models that aim to describe the data
in the simplest way, building in astrophyscially motivated priors, followed by sequentially adding complexities only as the data demands it. While in the current environment 
this simple approach is clearly more powerful, as we go forward into the data-rich catalog era flexible models can be a useful complement.
Non-parametric or semi-parametric methods are particularly useful at looking for unexpected features in the data, or finding where 
our simpler parametric models may be missing features in the data. Development and robust teting of these types of methods in the short term will provide
knock dowm effects later down the line as the size of our catalogs increase and we enter a data-dominated regime rather than prior/model dominated. 

The spin distributions of binary black holes (BBHs) have particular interest since the two main formation scenarios of BBHs produce 
distinctly different spin distrubitons \bruce{cite}. Isolated or field formation scenarios predict spins that are preferentially, although some small misalignment can occur 
depending on the nature of the supernovae kicks as each star collapses to a compact object \bruce{cite}. Dynamical formation scenarios occur in dense environments where 
many-body interactions between a large number compact objects occur before producing a binary that will merger within a Hubble time. Thus the dynamical channel produces 
an isotropic spin orientation distribution \bruce{cite}. It has also been of controversial interest in the literature recently, with different parameteric approaches finding 
different conclusions on the distribution of spins of BBHs. Recent studies have disagreed on the possible existence of a significant zero-spinning subpopulation, along with the 
presence of significant mis-alignment ($\cos{\theta_i} < 0.0$) \cite{o3b_astro_dist,RouletGWTC2Pop,BuildBetterSpinModels,GWTC3MonashSpin,Callister_NoEvidence}. A recent study 
has shown the spin orientation distribution infernce is senitive to modeling choices and may not peak at perfectly aligned spins, as is commonly assumed \cite{spinitasyoulike}. 
Other approaches to building better models for BBH spins again, as with the mass models, focus on iterating through different simple parameteric descriptions with 
increasing complexities \cite{BuildBetterSpinModels,Callister_NoEvidence,spinitasyoulike}. The spin distributions offer a nice use case for flexible data-driven models, both 
to help alleivate the current disputes in the literature and because there has been little work on flexible model development for the spin distribution of BBHs thus far.

Polynomial splines have been applied to success across different areas of gravitational-wave astronomy. They have been used to model the gravitational-wave data noise spectrum, 
detector calibration uncertainties, coherent gravitational waveform deviations, and modulations to a power law mass distribution \cite{Littenberg_2015,Edwards_2018,B_Farr_etal_2014,Edelman_2021,Edelman_2022ApJ}
In this paper we highlight how the use of basis-splines can provide a powerful non-parametric modeling approach to the astrophysical distributions of compact 
binaries. We illustrate how one can efficiently model the mass and spin distributions of compact binaries in GWTC-3 with basis splines to infer population properties using 
hierarchical Bayesian inference. We discuss our results in the context of current literature on compact object populations and how this method complements the simpler lower dimensional parametric models. 
This data-drien approach will be capable of finding more subtle features, that may not be expected, should they appear in future catalogs. 
In Section \ref{sec:basis_splines} we go over the background of basis splines, in Section \ref{sec:results} we present results using our basis 
spline model on the mass and spin distributions of BBHs in GWTC-3, followed by discussing results in their astrophysical context Section \ref{sec:astrodiscussion} 
and concluding in Section \ref{sec:conclusion}.
