\section{Introduction} \label{sec:intro}

Observations of gravitational waves (GWs) from compact binary coalescences (CBCs) are becoming a regular occurrence, 
recently producing catalogs of 90 such detections \citep{GWTC1,gwtc2,GWTC3}. With this growing catalog size, we are steadily approaching a more complete  
understanding of the population properties of CBCs, just 7 years after the first observation \citep{o1o2_pop,o3a_pop,o3b_astro_dist}. 
Due to their increased sensitivity, detections could occur as often as every day in the upcoming fourth observing run of the LIGO-Virgo-KAGRA (LVK) collaboration 
detectors \citep{aLIGO,aVIRGO,LVK_prospects}. This large influx of data will drastically improve our ability to study the population of observed CBCs by 
combining the insights from the entire catalog. The history of these dense objects is imprinted on the properties 
of the population  of CBCs, allowing us to learn about their formation and evolution from the last few seconds of gravitational radiation emitted \citep{Zevin_2017}. 
The population distributions of merging compact objects inferred from gravitational-wave observations have been shown to be particularly useful in order to
measure cosmological parameters \citep{Farr_2019HUB,gwtc3_cosmo}, constrain modified gravitational wave propagation \citep{ModGWProp,ModGWProp2}, 
constrain a running Planck mass \citep{Lagos_runningPlanckMass}, look for evidence of ultralight bosons through superradiance \citep{Ng_Boson2021,GWTC2_superradiance_Ng}, 
constrain stellar nuclear reaction rates \citep{Farmer_2019,Farmer_2020}, look for primordial black holes \citep{Ng_2021,KenNgPBH2022}, 
and to constrain physics of neutron stars \citep{Golomb_EOS,LandryRead_NS_Masses2021}. Through a better understanding of the mass, spin, and redshift distributions of 
compact binaries that will come with the increased catalog size, one can probe a wide range of different physical phenomena with even greater fidelity.

Studies using flexible methods to analyze the binary black hole (BBH) mass distribution with GWTC-2 found signs of structure beyond the first reported feature found at 
$m_1\sim35\,\msun$ \citep{Talbot_2018,o3a_pop,Tiwari_2021_b,Edelman_2022ApJ}. The evidence for this structure, 
such as the peak at $m_1\sim10\,\msun$, only grew after analyzing the latest catalog, GWTC-3, with the same models \citep{o3b_astro_dist,Tiwari_2022ApJ}, illustrating  
the usefulness of data-driven methods even with the relatively small current catalog size. Canonically, the approach to constructing population models has been 
to use simple parametric models that aim to describe the data in the simplest way, build in astrophysically motivated priors, then sequentially add in 
complexities only as the data demands. This simple approach is powerful in data-scarce, prior dominated environments; however, as we go forward into 
the data-rich catalog era, flexible methods such as non-parametric modeling will be necessary to discover new features in the data that can quickly 
become unwieldy to incorporate into the parametric approach. As opposed to parametric models, flexible and non-parametric models are data-driven and 
contribute little bias to functional form. Non-parametric or semi-parametric methods are particularly useful to search for unexpected features in the 
data, providing meaningful insight of these features that parametric models may fail to capture. Development and robust testing of such flexible methods in the short 
term will only provide knock on effects when the time comes to analyze the next catalog of CBC detections.

The spin distributions of BBHs are particularly interesting since the two main formation scenarios produce distinctly different spin distributions \bruce{cite}. 
Isolated or field formation scenarios predict component spins that are preferentially aligned with the binaries orbital angular momentum, although some small 
misalignment can occur depending on the nature of the supernova kicks as each star collapses to a compact object \bruce{cite}. Dynamical formation scenarios may occur in 
dense environments where many-body interactions between many compact objects cause a binary to harden so that it will merge within a Hubble time. Thus, the dynamical 
channel produces an isotropic spin tilt distribution \bruce{cite}. BBH spins have also been of controversial interest recently, with different parametric approaches to modeling 
the spin distribution coming to different conclusions. The studies have disagreed on the possible existence of a significant zero-spinning subpopulation, along with the presence of 
significant spin misalignment (i.e. $\cos{\theta_i} < 0.0$) \citep{o3b_astro_dist,RouletGWTC2Pop,BuildBetterSpinModels,GWTC3MonashSpin,Callister_NoEvidence}. 
Another study recently showed that the spin tilt distribution inference is sensitive to modeling choices and may not peak at perfectly aligned spins, as is often assumed \citep{spinitasyoulike}. 
Other approaches to building improved models for BBH spins again, as with the mass models, focus on iterating through different simple parametric descriptions with 
increasing complexities \citep{BuildBetterSpinModels,Callister_NoEvidence,spinitasyoulike}. The spin distributions offer a nice use case for flexible data-driven models, as 
we present in this manuscript, both to help alleviate the current disputes in the literature and because there has been little work on flexible model development for the 
spin distribution of BBHs beyond the aligned spin dimension thus far \citep{Tiwari_2021_a,Tiwari_2021_b}.

Polynomial splines have been applied to success across different areas of gravitational-wave astronomy. They have been used to model the gravitational-wave data noise spectrum, 
detector calibration uncertainties, coherent gravitational waveform deviations, and modulations to a power law mass distribution \citep{Littenberg_2015,Edwards_2018,B_Farr_etal_2014,Edelman_2021,Edelman_2022ApJ}
In this paper we highlight how the use of basis-splines can provide a powerful non-parametric modeling approach to the astrophysical distributions of compact 
binaries. We illustrate how one can efficiently model both the mass and spin distributions of merging compact binaries in GWTC-3 with basis splines to infer CBC population properties using 
hierarchical Bayesian inference. We discuss our results in the context of current literature on compact object populations and how this method complements the simpler lower 
dimensional parametric models in the short run, and will become necessary with future catalogs. Should they appear with more observations, this data-driven approach will provide checks of 
our understanding by uncovering more subtle -- potentially unexpected -- features. The rest of this manuscript is structured as follows: a description of the background of 
basis splines in section \ref{sec:basis_splines}, followed by a presentation of the results of our extensive, data-driven study of the mass and spin distributions of BBHs in GWTC-3 in section 
\ref{sec:results}. We then discuss these results and their astrophysical implications in section \ref{sec:astrodiscussion} and finish with our conclusions in section \ref{sec:conclusion}.