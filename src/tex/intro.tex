\section{Introduction} \label{sec:intro}

Observations of gravitational waves from compact binary coalescences (CBCs) is becoming a regular occurrence, 
recently producing catalogs of 90 such detections \bruce{cite GWTCs}. This growing catalog size is beginning to enable access to a more complete 
understanding of the population properties CBCs, which encodes the history of the formation and evolution of these extreme compact objects. 
In the coming fourth observing run of the LIGO-Virgo-KAGRA collaboration \bruce{cite detectors}, the catalog size is expected to increase by up to a factor of 5. 
Looking at population properties of compact objects lets us probe the complex physics common to all of our observations that governs the 
formation and evolution of merging compact objects. The history of these dense objects is imprinted on the properties of the population 
of CBCs, allowing us to learn about these not well understood ideas through careful evaluation of the enture distribution of observed GWs. 
The mass distribution of binary black holes has been used to measure cosmological parameters, constrain modified gravitational wave propagation, 
constrain a non-constant Planck mass, look for evidence of ultralight bosons through superradiance, put constraints on the uncertain nuclear reaction 
rates important for stellar evolution, look for primordial black holes, to constrain the neutron star equation of state, 
and to understand the fromation channels/environments of compact binary coalescences (CBCs). Through a better understanding of the mass, spin, 
and redshift distribution of compact binaries that will come with the increased catalog size, one can probe these dufferent physical phenoma with 
greater fidelity.

With the latest LIGO-Virgo-KAGRA (LVK) catalog of CBCs (GWTC-3), new understanding of the population has come into light. One of the main findings was that 
the mass distribution exhibits substructure with peaks rather than being perfectly described by a smooth power law distribution. The evidence for this was 
found with mutliple different flexible or non-parametric models that aim to be data-oriented with little bias on the function form. While these models are 
useful for looking for unexpected features in the data, or finding where our simpler models fail, they can be harder to interpret and are not based on specific
astrophysical motivation. These types of models will become more important as the size of our catalogs increase and we enter a data-dominated regime rather than
prior/model dominated. 

The spin distribution of binary black holes (BBHs) has also been of controversial interest in the literature recently, with different parameteric approaches finding 
rather different conclusions on the distribution of spins of BBHs which has significant impact on the inferences about relative contribution of dynamically formed
binaries versus ones formed in the field. This is another use case for flexible data-driven models to alleivate the discrepencies among parametric approaches. 
In particular there has been little work on flexible model development for the spin distribution of BBHs, which makes sense given the relatively smaller catalog 
sizes and the large uncertainty in individual black hole spin measurements with GWs.

In this paper we highlight how the use of basis-splines can be a useful model-agnostic modeling approach to the astrophysical distributions of compact 
binaries. We show that with a particular choice of basis, one can efficiently model the spin, mass and redshift distributions of compact binaries in GWTC-3 with
hierarchical Bayesian inference. We discuss our results in the context of current literature on compact object populations and illustrate how our results are similar
to those produced with the more simpler and astrophysically motivated parametric models. This data-drien approach will be capable of finding more subtle features, that 
may not be expected, should they appear in future catalogs. Additionally this method can be extended to higher-dimensional basis-splines in order to model 
joint population disributions over multiple surce parameters, which will be important to confidently disentangle different competing formation channels 
should they exist. In Section II we go over the background of basis splines and model construction. In Section III we present results using our basis spline model 
on the mass and spin distributions of BBHs in GWTC-3, followed by discussing the results and astrophysical implications in Section IV and concluding in Section V.
