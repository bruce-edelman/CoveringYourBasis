\section{Constructing A Basis} \label{sec:basis_splines}

A common non-parametric method in statistical applications is the use of basis splines. A spline function of order $k$, 
is a piece-wise polynomial of order $k$ polynomials stitched together from defined ``knot'' locations across the domain. 
They provide a useful and cheap way to interpolate generically smooth functions from a finite sampling of ``knot'' heights. 
Basis splines of order $k$ are a set of order $k$ polynomials that form a complete basis for any spline function of order $k$. 
Given an array of knot locations, $\mathbf{t}$ or knot vector, there exists a single unique linear combination of B-Splines for 
every possible spline function with knots, $\mathbf{t}$. One can construct the B-Splines basis components for a given knot vector 
using the Cox-de Boor recursion formula. Given knots $t_0$, $t_1$,...,$t_{i+k}$ we start with the base case as:

\begin{equation}
    B_{i,1}(x | \mathbf{t}) = 
    \begin{cases}
        1, & \text{if } t_i \leq x < t_{i+1} \\
        0, & \text{otherwise}
    \end{cases}
\end{equation}

\noindent combined with the recursion relation:

\begin{multline*}
    B_{i,k+1}(x | \mathbf{t}) = \omega_{i,k}(x | \mathbf{t})B_{i,k}(x | \mathbf{t})\\
                                + \big[1-\omega_{i+1,k}(x | \mathbf{t})\big] B_{i+1,k}(x | \mathbf{t})
\end{multline*}

\noindent where $\omega_{i,k}$ is defined as:

\begin{equation}
\omega_{i,k}(x | \mathbf{t}) =
\begin{cases}
    \frac{x-t_i}{t_{i+k}-t_i}, & t_{i+k} \neq t_i \\
    0, & \text{otherwise}
\end{cases}
\end{equation}

\begin{figure}[ht!]
    \script{spline_basis_plot.py}
        \includegraphics[width=0.45\textwidth]{figures/spline_basis_plot.pdf}
        \caption{Plot showing proper MSpline bases with 20 degrees of freedom and equal weights for each component.}
        \label{fig:spline_basis}
\end{figure}

\noindent allowing one to build a basis of $i$ components that spans the vector space of order-$k$ spline functions interpolated 
from this knot vector. This is known as the ``B-Spline'' basis. 

A similar basis, the M-Spline basis, has desirable properties when looking to model a probability density function. Namely, 
$M_i \geq 0$ within $(t_i, t_i+k)$, zero elsewhere, and normalized to unity, $\int M_i(x)dx = 1$. The M-Spline basis only differs 
from B-Splines by a normalization factor, namely:

\begin{equation}\label{eq:MB_SplineRelation}
M_{i,k} = \frac{k}{t_{i+k} - t_i} B_{i,k}
\end{equation}

\noindent We use this definition of the M-Spline basis for the models presented and applied in the later sections of this paper. 

\section{Model Construction}

\bruce{placeholder paragraph here to introduce our model constructions.}
We use hierarchical Bayesian inference to simultaneosly infer the mass, spin and redshift distributions of binary black holes (BBHs) using 
the 69 confident BBH detections reported in the recent LVK catalog of observations, GWTC-3. In particular we choose to parameterize the two
masses with the primary (more massive component) mass and the mass ratio defined as $q=\frac{m_2}{m_1}$. We choose to model 4 of the 6 total 
spin degrees of freedom of a binary merger with each component spin magnitude, and the cosine of the tilt angle which is defined as the angle 
between each spin vector and the binaries orbital angular mommentum vector. We additionally will fit a population model on the redshift distribution
of BBHs. We assume a $\Lambda\mathrm{CDM}$ cosmology defined by the Planck 2015 results, which defines an analytical mapping between each event's inferred
luminosity distance and it's redshift. While we choose to use MSpline distibutions for the masses and spins, we use the canonical \textsc{PowrerlawRedshift}
model describing the evolution of CBC merger rates as a power law in $(1+z)$ with a slope of $\lambda_z$. The corresponsing distribution is Thus
desribed by $p(z|\lambda)\propto \frac{dV_c}{dz} \frac{T_\mathrm{obs}}{1+z} (1+z)^\lambda$. We model the spin magnitude and tilt distibutions to be 
described by independent MSpline distributions for each component. 
