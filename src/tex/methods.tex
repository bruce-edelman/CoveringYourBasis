\section{Constructing A Basis} \label{sec:basis_splines}

A common non-parametric method used in many statistical applications is basis splines. A spline function of order $k$, 
is a piece-wise polynomial of order $k$ polynomials stitched together from defined ``knot'' locations across the domain. 
They provide a useful and cheap way to interpolate generically smooth functions from a finite sampling of ``knot'' heights. 
\it{Basis splines} (or ``B-Splines'') of order $k$ are a set of order $k$ polynomials that form a complete basis for any spline function of order $k$. 
Therefore, given an array of knot locations, $\mathbf{t}$ or knot vector, there exists a single unique linear combination of B-Splines for 
every possible spline function interpolated by knots, $\mathbf{t}$. The basis components for a given knot vector are constructed  
using the Cox-de Boor recursion formula \citep{deBoor78}. Given knots $t_0$, $t_1$,...,$t_{i+k}$ we start with the base case as:

\begin{figure}[ht!]
    \begin{centering}
        \includegraphics[width=\linewidth]{figures/spline_basis_plot.pdf}
        \caption{Plot showing a ``proper'' (see appendix \ref{sec:psplines}) M-Spline basis of order 3 (cubic) with 20 degrees of freedom and equal weights for each component. 
        In black, we show the resulting spline function with given equal weights and denote the location of the knots with gray x's.}
        \label{fig:spline_basis}
    \end{centering}
    \script{spline_basis_plot.py}
\end{figure}

\begin{equation}
    B_{i,1}(x | \mathbf{t}) = 
    \begin{cases}
        1, & \text{if } t_i \leq x < t_{i+1} \\
        0, & \text{otherwise}
    \end{cases}
\end{equation}

\noindent combined with the recursion relation:

\begin{multline*}
    B_{i,k+1}(x | \mathbf{t}) = \omega_{i,k}(x | \mathbf{t})B_{i,k}(x | \mathbf{t})\\
                                + \big[1-\omega_{i+1,k}(x | \mathbf{t})\big] B_{i+1,k}(x | \mathbf{t})
\end{multline*}

\noindent where above $\omega_{i,k}$ is defined as:

\begin{equation}
\omega_{i,k}(x | \mathbf{t}) =
\begin{cases}
    \frac{x-t_i}{t_{i+k}-t_i}, & t_{i+k} \neq t_i \\
    0, & \text{otherwise}
\end{cases}
\end{equation}

\noindent This is known as the ``B-Spline'' basis after it's inventor de Boor \citep{deBoor78}. The power of basis splines
comes from the fact that one only has to do the somewhat-expensive interpolation once for each set of points at which the spline is evaluated. 
This provides a considerable computational speedup since each evaluation of the spline function becomes a simpler operation that is just
a dot product of a matrix and a vector. This straightforward operation is also ideal for optimizations from the use GPU accelerators, 
enabling expensive Markov chain Monte Carlo (MCMC) based analyses with hundreds of parameters \bruce{citation?}. 
Basis splines can easily be generalized to their two-dimensional analog, producing tensor product basis splines that, 
with this computational advantage, allow for high fidelity modeling of two-dimensional spline functions.

A similar basis, the M-Spline basis \citep{monotone_regression_splines}, has desirable properties when looking to model a probability density function, 
which is the goal of population inference. Each basis component is scaled such that it is normalized over the domain, $\int M_i(x)dx = 1$. This ensures 
that given a vector of basis coefficients, if they sum to unity, the resulting spline function is normalized across the interpolation domain. 
The M-Spline basis differs from B-Splines by the scaling factor that normalizes each component:

\begin{equation}\label{eq:MB_SplineRelation}
M_{i,k} = \frac{k}{t_{i+k} - t_i} B_{i,k}
\end{equation}

\noindent We use this definition of the M-Spline basis for all the basis spline models used in this manuscript. Figure \ref{fig:spline_basis} 
shows an example of a cubic (3rd order) M-Spline basis with 20 degrees of freedom and knots linearly spaced from 0 to 1.
Another important feature of basis splines is that under appropriate prior conditions, one can alleviate sensitivities to arbitrarily 
chosen prior specifications that splines commonly struggle with. We look to the use of penalized splines (or P-Splines) \citep{eilers2021practical,BayesianPSplines,Jullion2007RobustSO}, 
where one adds a smoothing prior based on the difference of basis spline coefficients. This allows one to populate the domain with more dense knots 
without the worry of extra variance in the inferred spline functions. We discuss the details of our smoothing prior implementation 
in more detail in Appendix \ref{sec:psplines}, along with our specific prior and basis choices for each model in Appendix \ref{sec:modelpriors}.
