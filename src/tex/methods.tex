\section{Constructing A Basis} \label{sec:basis_splines}

\begin{figure}[ht!]
    \script{spline_basis_plot.py}
        \includegraphics[width=0.45\textwidth]{figures/spline_basis_plot.pdf}
        \caption{Plot showing a ``proper'' MSpline basis of order 3 (cubic) with 20 degrees of freedom and equal weights for each component. 
        In black we show the resulting spline function with given equal weights and denote the location of the knots with grey x's.}
        \label{fig:spline_basis}
\end{figure}

A common non-parametric method in used in many statistical applications is basis splines. A spline function of order $k$, 
is a piece-wise polynomial of order $k$ polynomials stitched together from defined ``knot'' locations across the domain. 
They provide a useful and cheap way to interpolate generically smooth functions from a finite sampling of ``knot'' heights. 
Basis splines of order $k$ are a set of order $k$ polynomials that form a complete basis for any spline function of order $k$. 
Given an array of knot locations, $\mathbf{t}$ or knot vector, there exists a single unique linear combination of B-Splines for 
every possible spline function with knots, $\mathbf{t}$. One can construct the B-Splines basis components for a given knot vector 
using the Cox-de Boor recursion formula \cite{deBoor78}. Given knots $t_0$, $t_1$,...,$t_{i+k}$ we start with the base case as:

\begin{equation}
    B_{i,1}(x | \mathbf{t}) = 
    \begin{cases}
        1, & \text{if } t_i \leq x < t_{i+1} \\
        0, & \text{otherwise}
    \end{cases}
\end{equation}

\noindent combined with the recursion relation:

\begin{multline*}
    B_{i,k+1}(x | \mathbf{t}) = \omega_{i,k}(x | \mathbf{t})B_{i,k}(x | \mathbf{t})\\
                                + \big[1-\omega_{i+1,k}(x | \mathbf{t})\big] B_{i+1,k}(x | \mathbf{t})
\end{multline*}

\noindent where $\omega_{i,k}$ is defined as:

\begin{equation}
\omega_{i,k}(x | \mathbf{t}) =
\begin{cases}
    \frac{x-t_i}{t_{i+k}-t_i}, & t_{i+k} \neq t_i \\
    0, & \text{otherwise}
\end{cases}
\end{equation}

\noindent allowing one to build a basis of $i$ components that spans the vector space of order-$k$ spline functions interpolated 
from this knot vector. This is known as the ``B-Spline'' basis after it's inventor de Boor \cite{deBoor78}. The power of basis splines
comes from the fact that you only have to do the, somewhat expensive, interpolation once for each set of points you want to evaluate your
spline functions at. This allows for much quicker speeds since each evaluation of the spline function becomes a simpler operation that is just
a dot product of 2d matrix and 1d vector. This straightforward operation is also ideal for optimizations from the use GPU accelerators, 
enabling expensive markov chain monte carlo (MCMC) based analyses with hundreds of parameters. Basis splines can easily be generalized to 
their two-dimensional analog, producing tensor product basis splines that, with this computational advantange, allow for high fidelity
modeling of two-dimensional spline functions.

A similar basis, the M-Spline basis \citep{monotone_regression_splines}, has desirable properties when looking to model a probability density function, 
which is the goal of population inference. Namely, $M_i \geq 0$ within $(t_i, t_i+k)$, zero elsewhere, and normalized to unity, $\int M_i(x)dx = 1$. We now have
a useful basis that can represent all spline functions that are probability density functions given our array of knots. The M-Spline basis differs 
from B-Splines by the normalization factor:

\begin{equation}\label{eq:MB_SplineRelation}
M_{i,k} = \frac{k}{t_{i+k} - t_i} B_{i,k}
\end{equation}

\noindent We use this definition of the M-Spline basis for the models presented and applied in the later sections of this paper. 
Figure \ref{fig:spline_basis} shows an example of a 3rd order M-Spline basis with 20 degrees of freedom and knots linearly spaced from 0 to 1.
Another important feature of basis splines is that under appropiate prior conditions, one can alleviate sensitivities to arbitrarily 
chosen prior specifications that splines commonly struggle with. We look to the use of penalized splines or P-Splines \cite{eilers2021practical,BayesianPSplines,Jullion2007RobustSO}, 
that adds a smoothing penalty (or prior) based on the difference of basis spline coeficients. This allows one to much more denseley populate
the domain with knots without the worry of extra variance in the inferred spline functions. We discuss the smoothing prior in more detail 
in Appendix \ref{sec:psplines}, along with our specific prior and basis choices for each model in Appendix \ref{sec:modelpriors}.