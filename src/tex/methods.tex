\section{Building the Model} \label{sec:methods}

We parameterize the binaries' masses with the primary (defined as the more massive component) mass 
and the mass ratio -- defined as $q=\frac{m_2}{m_1}$, bounded in the range from 0 to 1. Furthermore, we model 4 of the 6 total 
spin degrees of freedom of a binary merger with each component spin magnitude $a_1$ and $a_2$, along with the cosine of the tilt angle of each component, 
$\cos{\theta_1}$ and $\cos{\theta_2}$. The tilt angle is defined as the angle between each components' spin vector and the binaries orbital angular momentum vector. 
We assume the polar spin angles are uniformly distributed in the orbital plane. While we choose to use M-Spline distributions to model the primary mass and spin distributions, we 
model the mass ratio distribution with a power law, i.e. $p(q | m_1, m_\mathrm{min}, \beta) \propto q^{\beta} \Theta(qm_1 - m_\mathrm{min}) \Theta(m_1 - qm_1)$, where $\beta$ is 
the power law index and $\Theta$ is the Heaviside step function which ensures $m_2$ is within the allowed range, [$m_\mathrm{min}$, $m_1$] \citep{Talbot_2018,o1o2_pop,o3a_pop}.
Additionally, we fit a population model on the redshift or luminosity distance distribution of BBHs, assuming a $\Lambda\mathrm{CDM}$ cosmology defined by the parameters 
from the Planck 2015 results \citep{Planck2015}. This defines an analytical mapping between each event's inferred luminosity distance, and it's redshift, which we now use interchangeably. 
We model the evolution of the merger rate with redshift with the \textsc{PowerlawRedshift} model, described by $p(z|\lambda_z)\propto \frac{dV_c}{dz}(1+z)^{\lambda_z-1}$ \citep{Fishbach_2018redshift}. 
We describe our specific prior choices on the hyperparameters and each model in more detail in Appendix \ref{sec:modelpriors}.