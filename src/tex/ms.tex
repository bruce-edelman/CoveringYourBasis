\documentclass[twocolumn, linenumber]{aastex63}

\newcommand{\msun}{\ensuremath{{\rm M}_\odot}}
\newcommand{\result}[1]{\textcolor{red}{#1}}
\newcommand{\bruce}[1]{\textcolor{blue}{BE: #1}}
\newcommand{\NewChange}[1]{\textcolor{green}{#1}}

\graphicspath{{./}{figures/}}

\usepackage[perpage]{footmisc}
\usepackage{colortbl}
\usepackage{lineno}
\usepackage{amsmath}
\usepackage{appendix}
\definecolor{Gray}{gray}{0.9}
\linenumbers

\begin{document}

\title{Covering Your Basis: Model-Agnostic Inference of LIGO-Virgos Binary Black Hole Spin Distribution}

\author{Bruce Edelman}
\email{bedelman@uoregon.edu}
\affiliation{Institute  for  Fundamental  Science, Department of Physics, University of Oregon, Eugene, OR 97403, USA}

\author{Lexi Vives}
\affiliation{Institute  for  Fundamental  Science, Department of Physics, University of Oregon, Eugene, OR 97403, USA}

\author{Ben Farr}
\affiliation{Institute  for  Fundamental  Science, Department of Physics, University of Oregon, Eugene, OR 97403, USA}

\begin{abstract}
We introduce a non-parametric model built from basis splines, and apply it to infer the astrophysical mass distribution of binary black holes using GWTC-3.
\end{abstract}


\section{Introduction} \label{sec:intro}

Now that gravitational wave observations of compact binary coalescence is a regular occurrence, producing catalogs of nearly 100 such detections, we now have access to a complete understanding of the population of compact binaries. In the coming fourth observing run of the LIGO-Virgo-KAGRA collaboration, the catalog size is expected to increase by a factor of 5. The extensive physics governing the formation and evolution of compact objects that end up merging is imprinted on the properties of the population of CBCs, allowing us to learn about these not well understood ideas through looking at the distribution of observed GWs. In particular the mass distribution of binary black holes has been shown to be very useful in measuring cosmological parameters, constraining modified gravitational wave propagation, constraining a running Planck mass, looking of evidence of ultralight bosons through superradiance, putting constraints on uncertain nuclear reaction rates important for stellar evolution, and even looking for primordial black holes. Through a better understanding of the mass, spin, and redshift distribution of compact binaries we can extend this be a useful tool in all the different physical applications mentioned above. In this paper we highlight how the use of basis-splines can be a useful model-agnostic modeling approach to the astrophysical distributions of compact binaries. We show that with a particular choice of basis, we can efficiently model the spin and/or mass distribution of compact binaries in GWTC-3 with spline regression in hierarchical Bayesian inference. We show our results are similar to those produced with the more simpler and motivated parametric models, while being capable of finding more subtle features should they appear in future catalogs. This method can be extended to higher-dimensional basis-splines to model correlations between different distributions which will be necessary to confidently disentangle differently competing formation channels should they exist. 

\section{Methods} \label{sec:methods}
\subsection{Constructing A Basis} \label{sec:basis_splines}

A common non-parametric method in statistical applications is the use of B-Splines or basis splines. A spline function of order $k$, is a piece-wise polynomial of order $k-1$ polynomials stitched together from defined ``knot'' locations across the domain. They provide a useful and cheap way to interpolate generically smooth functions from a finite sampling of ``knot'' heights. B-Splines of order $k$ are a set of order $k-1$ polynomials that form a complete basis for any spline function of order $k$. Given an array of knot locations, $\mathbf{t}$ or knot vector, there exists a single unique linear combination of B-Splines for every possible spline function with knots, $\mathbf{t}$. One can construct the B-Splines for a given knot vector using the Cox-de Boor recursion formula -- for knots $t_0$, $t_1$,...,$t_{i+k}$ the basis components are given by:
\begin{equation}
    B_{i,1}(x | \mathbf{t}) = 
    \begin{cases}
        1, & \text{if } t_i \leq x < t_{i+1} \\
        0, & \text{otherwise}
    \end{cases}
\end{equation}
\begin{equation}
    B_{i,k+1}(x | \mathbf{t}) = \omega_{i,k}(x)B_{i,k}(x) + \big[1-\omega_{i+1,k}(x)\big]B_{i+1,k}(x)
\end{equation}
\noindent Where $\omega_{i,k}$ above is defined as:
\begin{equation}
    \omega_{i,k}(x | \mathbf{t}) =
    \begin{cases}
        \frac{x-t_i}{t_{i+k}-t_i}, & t_{i+k} \neq t_i \\
        0, & \text{otherwise}
    \end{cases}
\end{equation}

A similar basis, the M-Spline basis, has desirable properties when looking to model a probability density function. Namely, $M_i \geq 0$ within $(t_i, t_i+k)$, zero elsewhere, and normalized to unity, $\int M_i(x)dx = 1$. The M-Spline basis only differs from B-Splines by a normalization factor, namely:
\begin{equation}\label{eq:MB_SplineRelation}
    M_{i,k} = \frac{k}{t_{i+k} - t_i} B_{i,k}
\end{equation}
\noindent We use this definition of the M-Spline basis for the models presented and applied in the later sections of this paper. 

Figure shows examples of two different M-Spline bases with 10 degrees of freedom and knots spaced linearly from 0 to 1. Notice that when we introduced knots and in equation \ref{eq:MB_SplineRelation}, you can see there needs to be knots up to $t_{i+k}$ while $i$ indexes the desired degrees of freedom in your basis. This means in practice one must define $n+k$ knots. However, you have control in the boundary conditions by how many of the $n+k$ knots you place at the lower and upper bounds. By placing $k$ knots at each boundary (leaving $n-k$ interior knots), the first and last basis components are maximized at the boundaries. Alternatively one may want to a-priori enforce vanishing boundary conditions, for example, if they knew for a fact that the function they are modeling has $p(x_\mathrm{bound}) = 0$.  This can be done by placing $k-1$ knots at the boundaries you wish the spline to vanish at. See figure for an example of a cubic M-Spline (k=3) basis with 10 degrees of freedom with $k=3$ knots at each boundary (left) versus enforcing vanishing boundary conditions by having $k-1=2$ knots at each boundary (right).



\subsection{Parameterized Perturbations} \label{sec:perturbation_model}
\subsection{Generic Mass Model} \label{sec:nonparametric_model}
\section{Results} \label{sec:results}
\subsection{Mass Spectrum Inference with GWTC-3} \label{sec:GWTC3_inferece}
\section{Conclusions}\label{sec:conclusion}
\section{Acknowledgements}\label{sec:acknowledments}
This research has made use of data, software and/or web tools obtained from the Gravitational Wave Open Science Center (\url{https://www.gw-openscience.org/}), a service of LIGO Laboratory, the LIGO Scientific Collaboration and the Virgo Collaboration. This work benefited from access to the University of Oregon high performance computer, Talapas.  This material is based upon work supported in part by the National Science Foundation under Grant PHY-1807046 and work supported by NSF’s LIGO Laboratory which is a major facility fully funded by the National Science Foundation.
\software{
\textsc{Astropy}~\citep{2018AJ....156..123A},
\textsc{NumPy}~\citep{harris2020array},
\textsc{Matplotlib}~\citep{Hunter:2007},
\textsc{bilby}~\citep{Ashton_2019},
\textsc{GWPopulation}~\citep{Talbot_2019}
}
\bibliography{references}{}
\bibliographystyle{aasjournal}

\appendix
\section{Hierarchical Bayesian Inference} \label{sec:hierarchical_inference}

% This appendix goes over in detail the hierarchical bayeisan inference framework
We use hierarchical Bayesian inference to simultaneously infer the population distributions of the primary masses ($m_1$), mass ratios ($q$) and the redshifts ($z$) of observed BBHs. For a set of hyper-parameters, $\Lambda$, and local ($z=0$) merger rate density (units of mergers per co-moving volume per time), $\mathcal{R}_0$, we write the overall number density of BBH mergers in the universe as: 

\begin{equation} \label{number_density}
     \frac{dN(m_1, q, z | \mathcal{R}_0, \Lambda)}{dm_1dqdz} = \frac{dV_c}{dz}\bigg(\frac{T_\mathrm{obs}}{1+z}\bigg) \frac{d\mathcal{R}(m_1, q, z | \mathcal{R}_0, \Lambda)}{dm_1dq} = \mathcal{R}_0 p(m_1 | \Lambda) p(q | m_1, \Lambda) p(z | \Lambda)
\end{equation}

\noindent
Above, $dV_c$ is the co-moving volume element \citep{hogg_cosmo} and $T_\mathrm{obs}$, the observing time period that produced the catalog with the related factor of $1+z$ converting this detector-frame time to source-frame. We assume a LambdaCDM cosmology using the cosmological parameters from \citet{Planck2015}. The redshift evolution of the merger rate follows $p(z|\lambda) \propto \frac{dV_c}{dz}\frac{1}{1+z}(1+z)^\lambda$. Integrating equation \ref{number_density} across all masses, and up to some redshift, $z_m$, returns the total number of BBH mergers in the universe out to that redshift. Let $\{d_i\}$ represent a set of data from $N_\mathrm{obs}$ observed gravitational waves associated with BBH mergers. We model the merger rate as an inhomogenous point process and when imposing a log-uniform prior on the merger rate, we can marginalize over the merger rate to get the posterior distribution of our hyper-parameters, $\Lambda$ \citep{Mandel_2019, Vitale_2021}.

\begin{equation} \label{posterior}
    p\left(\Lambda | \{d_i\} \right) \propto \frac{p(\Lambda)}{\xi(\Lambda)^{N_\mathrm{obs}}} \prod_{i=1}^{N_\mathrm{obs}} \Bigg[ \int \mathcal{L}\left(d_i | m_1^i, q^i, z^i \right) p(m_1 | \Lambda) p(q | m_1, \Lambda) p(z | \Lambda) dm_1 dq dz \Bigg],
\end{equation}

\noindent
where, $\mathcal{L}(d_i|m_1, q, z)$, is the single-event likelihood function from each events original analysis, and $\xi(\Lambda)$ is the detection efficiency given a population distribution described by $\Lambda$. The procedure for calculating $\xi(\Lambda)$ is described in more detail below. The LVC reports out posterior samples for each observed event, with which we can use importance sampling to estimate the integrals above in equation \ref{posterior}. We replace the integrals with ensemble averages over $K_i$ posterior samples associated with each event in the catalog:

\begin{equation}
 p\left(\Lambda | \{d_i\}\right) \propto p(\Lambda) \prod_{i=1}^{N_\mathrm{obs}} \bigg[ \frac{1}{K_i} \sum_{j=1}^{K_i} \frac{p(m_1^{i,j}|\Lambda)p(q^{i,j}|\Lambda)p(z^{i,j}|\Lambda)}{\pi(m_1^{i,j}, q^{i,j}, z^{i,j})} \bigg]
\end{equation}

\noindent 
Here $j$ indexes the $K_i$ posterior samples from each event and $\pi(m_1, q, z)$ is the default prior used by parameter estimations that produced the posterior samples for each event. In the analyses of GWTC-2 the default prior used is uniform in detector frame masses and Euclidean volume. The corresponding prior evaluated in source frame masses and redshift is $\pi(m_1, q, z) \propto m_1(1+z)^2 D_L^2(z) \frac{dD_L}{dz}$, where $D_L$ is the luminosity distance. 

To carefully incorporate selection effects to our model we need to quantify the detection efficiency, $\xi(\Lambda)$, of the search pipelines that were used to create GWTC-2, at a given population distribution described by $\Lambda$.
 
\begin{equation} \label{det_eff}
     \xi(\Lambda) = \int dm_1 dq dz P_\mathrm{det}(m_1, q, z)p(m_1 | \Lambda) p(q | m_1, \Lambda) p(z | \Lambda)
\end{equation}
 
\noindent The above integral is not tractable since there is no analytic prescription for $P_\mathrm{det}(m_1,q,z)$, the detection probability of an individual event. To estimate this integral we use a software injection campaign where GWs from a fixed, broad population of sources are simulated, put into real detector data, and then run through the same search pipelines that were used to produce the catalog we are analyzing \footnote{For O3a we used the injection sets used by \citet{o3a_pop}, which can be found at https://dcc.ligo.org/LIGO-P2000217/public. For O1/O2 we used the mock injection sets used by \citet{o1o2_pop} which can be found at https://dcc.ligo.org/LIGO-P2000434/public}. With these search results in hand, we use importance sampling to evaluate the integral in equation \ref{det_eff}:

\begin{equation} \label{xi}
     \xi(\Lambda) = \frac{1}{N_\mathrm{inj}} \sum_{i=1}^{N_\mathrm{found}} \frac{p(m_1^i | \Lambda) p(q^i | m_1, \Lambda) p(z^i | \Lambda)}{p_\mathrm{inj}(m_1^i, q^i, z^i)}
\end{equation}

\noindent
Where the sum indexes only over the $N_\mathrm{found}$ injections that were successfully detected out of $N_\mathrm{inj}$ total injections, and $p_\mathrm{inj}(m_1, q, z)$ is the reference distribution from which the injections were drawn. Additionally, we follow the procedure outlined in \citet{Farr_2019} to marginalize the uncertainty in our estimate of $\xi(\Lambda)$ due to a finite number of simulated events. We make the assumption that repeated sampling of $\xi(\Lambda)$ will follow a normal distribution with $\xi(\Lambda) \sim \mathcal{N}(\mu(\Lambda), \sigma(\Lambda))$, where the mean, $\mu$, is the estimate from equation \ref{xi}, while the variance, $\sigma^2$, is defined as:

\begin{equation}
    \sigma^2(\Lambda) \equiv \frac{\mu^2(\Lambda)}{N_\mathrm{eff}} \simeq \frac{1}{N^2_\mathrm{inj}} \sum_{i=1}^{N_\mathrm{found}} \bigg[\frac{p(m_1 | \Lambda) p(q | m_1, \Lambda) p(z | \Lambda)}{p_\mathrm{inj}(m_1, q, z)}\bigg]^2 - \frac{\mu^2(\Lambda)}{N_\mathrm{inj}}
\end{equation}

\noindent
Above we define $N_\mathrm{eff}$ as the effective number of independent draws contributing to the importance sampled estimate, in which we verify to be sufficiently high after re-weighting the injections to a given population (i.e. $N_\mathrm{eff} > 4N_\mathrm{obs}$). We write the hyper-posterior marginalized over the merger rate and uncertainty in estimation of $\xi(\Lambda)$, neglecting terms of $\mathcal{O}(N_\mathrm{eff}^{-2})$ or greater \citep{Farr_2019}, as:

\begin{equation}\label{importance-posterior}
    \log p\left(\Lambda | \{d_i\}\right) \propto \sum_{i=1}^{N_\mathrm{obs}} \log \bigg[ \frac{1}{K_i} \sum_{j=1}^{K_i} \frac{p(m_1^{i,j}|\Lambda)p(q^{i,j}|\Lambda)p(z^{i,j}|\Lambda)}{\pi(m_1^{i,j}, q^{i,j}, z^{i,j})} \bigg] -  \\
    N_\mathrm{obs} \log \mu + \frac{3N_\mathrm{obs} + N_\mathrm{obs}^2}{2N_\mathrm{eff}} + \mathcal{O}(N_\mathrm{eff}^{-2}).
\end{equation}

\noindent
We explicitly enumerate each of the models used in this work for $p(m_1|\Lambda)$, $p(q|m_1, \Lambda)$, and $p(z|\Lambda)$ along with their respective hyper-parameters and prior distributions in Table \ref{tab:model_params}. To calculate marginal likelihoods and draw samples of the hyper parameters from the hierarchical posterior distribution shown in equation \ref{importance-posterior}, we use the \textsc{Bilby} \citep{Ashton_2019, bilby_gwtc1} and \textsc{GWPopulation} \citep{Talbot_2019} Bayesian inference software libraries with the \textsc{Dynesty} dynamic nested sampling algorithm \citep{Speagle_2020}.


\section{Validation Studies} \label{sec:validation}

\end{document}