\documentclass[twocolumn, linenumbers]{aastex631}

\newcommand{\msun}{\ensuremath{{\rm M}_\odot}}
\newcommand{\result}[1]{\textcolor{red}{#1}}
\newcommand{\bruce}[1]{\textcolor{blue}{BE: #1}}
\newcommand{\NewChange}[1]{\textcolor{green}{#1}}

\newcommand{\CIPlusMinus}[1]{{#1[median]^{+#1[error plus]}_{-#1[error minus]}}}
\newcommand{\CIBoundsBracket}[1]{{[#1[5th percentile], #1[95th percentile]]}}
\newcommand{\CIBoundsDash}[1]{{#1[5th percentile]\textendash#1[95th percentile]}}

\graphicspath{{./}{figures/}}

\usepackage{showyourwork}
\usepackage{bm}
\usepackage{amsmath}
\usepackage{appendix}
\makeatletter\newcommand\macros[1][all]{\ifnum\pdfstrcmp{#1}{all}=0\def\macros@out{\{"MSplineIndependentCompSpins": \{"peakCosTilt1": \{"median": "0.19", "error plus": "0.73", "error minus": "0.73", "5th percentile": "{-}0.54", "95th percentile": "0.89"\}, "peakCosTilt2": \{"median": "0.40", "error plus": "0.64", "error minus": "1.06", "5th percentile": "{-}0.66", "95th percentile": "1.00"\}, "log10gammaFrac1": \{"median": "0.13", "error plus": "0.71", "error minus": "0.70", "5th percentile": "{-}0.57", "95th percentile": "0.84"\}, "log10gammaFrac2": \{"median": "0.28", "error plus": "0.72", "error minus": "0.74", "5th percentile": "{-}0.46", "95th percentile": "0.99"\}, "negFrac1": \{"median": "0.43", "error plus": "0.19", "error minus": "0.18", "5th percentile": "0.25", "95th percentile": "0.62"\}, "negFrac2": \{"median": "0.37", "error plus": "0.19", "error minus": "0.16", "5th percentile": "0.21", "95th percentile": "0.57"\}\}, "MSplineIIDCompSpins": \{"peakCosTilt": \{"median": "0.37", "error plus": "0.64", "error minus": "0.50", "5th percentile": "{-}0.14", "95th percentile": "1.00"\}, "log10gammaFrac": \{"median": "0.31", "error plus": "0.66", "error minus": "0.66", "5th percentile": "{-}0.35", "95th percentile": "0.96"\}, "negFrac": \{"median": "0.34", "error plus": "0.12", "error minus": "0.12", "5th percentile": "0.23", "95th percentile": "0.47"\}\}, "ChiEffective": \{"default": \{"FracBelowNeg0p3": \{"median": "0.01", "error plus": "0.02", "error minus": "0.01", "5th percentile": "0.00", "95th percentile": "0.03"\}, "FracBelow0": \{"median": "0.43", "error plus": "0.08", "error minus": "0.15", "5th percentile": "0.28", "95th percentile": "0.49"\}, "PeakChiEff": \{"median": "0.02", "error plus": "0.03", "error minus": "0.02", "5th percentile": "{-}0.01", "95th percentile": "0.05"\}\}, "iid": \{"FracBelowNeg0p3": \{"median": "0.02", "error plus": "0.03", "error minus": "0.02", "5th percentile": "0.00", "95th percentile": "0.05"\}, "FracBelow0": \{"median": "0.35", "error plus": "0.10", "error minus": "0.12", "5th percentile": "0.23", "95th percentile": "0.44"\}, "PeakChiEff": \{"median": "0.03", "error plus": "0.04", "error minus": "0.02", "5th percentile": "0.01", "95th percentile": "0.07"\}\}, "ind": \{"FracBelowNeg0p3": \{"median": "0.03", "error plus": "0.02", "error minus": "0.01", "5th percentile": "0.02", "95th percentile": "0.05"\}, "FracBelow0": \{"median": "0.39", "error plus": "0.08", "error minus": "0.07", "5th percentile": "0.32", "95th percentile": "0.47"\}, "PeakChiEff": \{"median": "0.03", "error plus": "0.03", "error minus": "0.03", "5th percentile": "0.00", "95th percentile": "0.05"\}\}, "chieff": \{"FracBelowNeg0p3": \{"median": "0.07", "error plus": "0.05", "error minus": "0.03", "5th percentile": "0.04", "95th percentile": "0.12"\}, "FracBelow0": \{"median": "0.40", "error plus": "0.10", "error minus": "0.10", "5th percentile": "0.29", "95th percentile": "0.50"\}, "PeakChiEff": \{"median": "0.04", "error plus": "0.04", "error minus": "0.04", "5th percentile": "{-}0.00", "95th percentile": "0.07"\}\}\}, "MassDistribution": \{"PLPeak": \{"m\_1percentile": \{"median": "6.32", "error plus": "0.64", "error minus": "1.08", "5th percentile": "5.24", "95th percentile": "6.90"\}, "m\_99percentile": \{"median": "44.77", "error plus": "9.94", "error minus": "6.18", "5th percentile": "38.59", "95th percentile": "55.46"\}\}, "MSpline": \{"m\_1percentile": \{"median": "6.01", "error plus": "0.51", "error minus": "0.19", "5th percentile": "5.82", "95th percentile": "6.59"\}, "m\_99percentile": \{"median": "80.10", "error plus": "11.37", "error minus": "12.53", "5th percentile": "67.57", "95th percentile": "91.26"\}\}, "PLSpline": \{"m\_1percentile": \{"median": "5.83", "error plus": "1.03", "error minus": "1.47", "5th percentile": "4.35", "95th percentile": "6.81"\}, "m\_99percentile": \{"median": "43.20", "error plus": "10.47", "error minus": "4.71", "5th percentile": "38.49", "95th percentile": "54.97"\}\}\}\}}\else\ifnum\pdfstrcmp{#1}{MSplineIndependentCompSpins}=0\let\macros@out\macros@I\else\ifnum\pdfstrcmp{#1}{MSplineIIDCompSpins}=0\let\macros@out\macros@II\else\ifnum\pdfstrcmp{#1}{ChiEffective}=0\let\macros@out\macros@III\else\ifnum\pdfstrcmp{#1}{MassDistribution}=0\let\macros@out\macros@IV\else\def\macros@out{??}\fi\fi\fi\fi\fi\macros@out}\newcommand\macros@I[1][all]{\ifnum\pdfstrcmp{#1}{all}=0\def\macros@I@out{\{"peakCosTilt1": \{"median": "0.19", "error plus": "0.73", "error minus": "0.73", "5th percentile": "{-}0.54", "95th percentile": "0.89"\}, "peakCosTilt2": \{"median": "0.40", "error plus": "0.64", "error minus": "1.06", "5th percentile": "{-}0.66", "95th percentile": "1.00"\}, "log10gammaFrac1": \{"median": "0.13", "error plus": "0.71", "error minus": "0.70", "5th percentile": "{-}0.57", "95th percentile": "0.84"\}, "log10gammaFrac2": \{"median": "0.28", "error plus": "0.72", "error minus": "0.74", "5th percentile": "{-}0.46", "95th percentile": "0.99"\}, "negFrac1": \{"median": "0.43", "error plus": "0.19", "error minus": "0.18", "5th percentile": "0.25", "95th percentile": "0.62"\}, "negFrac2": \{"median": "0.37", "error plus": "0.19", "error minus": "0.16", "5th percentile": "0.21", "95th percentile": "0.57"\}\}}\else\ifnum\pdfstrcmp{#1}{peakCosTilt1}=0\let\macros@I@out\macros@V\else\ifnum\pdfstrcmp{#1}{peakCosTilt2}=0\let\macros@I@out\macros@VI\else\ifnum\pdfstrcmp{#1}{log10gammaFrac1}=0\let\macros@I@out\macros@VII\else\ifnum\pdfstrcmp{#1}{log10gammaFrac2}=0\let\macros@I@out\macros@VIII\else\ifnum\pdfstrcmp{#1}{negFrac1}=0\let\macros@I@out\macros@IX\else\ifnum\pdfstrcmp{#1}{negFrac2}=0\let\macros@I@out\macros@X\else\def\macros@I@out{??}\fi\fi\fi\fi\fi\fi\fi\macros@I@out}\newcommand\macros@II[1][all]{\ifnum\pdfstrcmp{#1}{all}=0\def\macros@II@out{\{"peakCosTilt": \{"median": "0.37", "error plus": "0.64", "error minus": "0.50", "5th percentile": "{-}0.14", "95th percentile": "1.00"\}, "log10gammaFrac": \{"median": "0.31", "error plus": "0.66", "error minus": "0.66", "5th percentile": "{-}0.35", "95th percentile": "0.96"\}, "negFrac": \{"median": "0.34", "error plus": "0.12", "error minus": "0.12", "5th percentile": "0.23", "95th percentile": "0.47"\}\}}\else\ifnum\pdfstrcmp{#1}{peakCosTilt}=0\let\macros@II@out\macros@XI\else\ifnum\pdfstrcmp{#1}{log10gammaFrac}=0\let\macros@II@out\macros@XII\else\ifnum\pdfstrcmp{#1}{negFrac}=0\let\macros@II@out\macros@XIII\else\def\macros@II@out{??}\fi\fi\fi\fi\macros@II@out}\newcommand\macros@III[1][all]{\ifnum\pdfstrcmp{#1}{all}=0\def\macros@III@out{\{"default": \{"FracBelowNeg0p3": \{"median": "0.01", "error plus": "0.02", "error minus": "0.01", "5th percentile": "0.00", "95th percentile": "0.03"\}, "FracBelow0": \{"median": "0.43", "error plus": "0.08", "error minus": "0.15", "5th percentile": "0.28", "95th percentile": "0.49"\}, "PeakChiEff": \{"median": "0.02", "error plus": "0.03", "error minus": "0.02", "5th percentile": "{-}0.01", "95th percentile": "0.05"\}\}, "iid": \{"FracBelowNeg0p3": \{"median": "0.02", "error plus": "0.03", "error minus": "0.02", "5th percentile": "0.00", "95th percentile": "0.05"\}, "FracBelow0": \{"median": "0.35", "error plus": "0.10", "error minus": "0.12", "5th percentile": "0.23", "95th percentile": "0.44"\}, "PeakChiEff": \{"median": "0.03", "error plus": "0.04", "error minus": "0.02", "5th percentile": "0.01", "95th percentile": "0.07"\}\}, "ind": \{"FracBelowNeg0p3": \{"median": "0.03", "error plus": "0.02", "error minus": "0.01", "5th percentile": "0.02", "95th percentile": "0.05"\}, "FracBelow0": \{"median": "0.39", "error plus": "0.08", "error minus": "0.07", "5th percentile": "0.32", "95th percentile": "0.47"\}, "PeakChiEff": \{"median": "0.03", "error plus": "0.03", "error minus": "0.03", "5th percentile": "0.00", "95th percentile": "0.05"\}\}, "chieff": \{"FracBelowNeg0p3": \{"median": "0.07", "error plus": "0.05", "error minus": "0.03", "5th percentile": "0.04", "95th percentile": "0.12"\}, "FracBelow0": \{"median": "0.40", "error plus": "0.10", "error minus": "0.10", "5th percentile": "0.29", "95th percentile": "0.50"\}, "PeakChiEff": \{"median": "0.04", "error plus": "0.04", "error minus": "0.04", "5th percentile": "{-}0.00", "95th percentile": "0.07"\}\}\}}\else\ifnum\pdfstrcmp{#1}{default}=0\let\macros@III@out\macros@XIV\else\ifnum\pdfstrcmp{#1}{iid}=0\let\macros@III@out\macros@XV\else\ifnum\pdfstrcmp{#1}{ind}=0\let\macros@III@out\macros@XVI\else\ifnum\pdfstrcmp{#1}{chieff}=0\let\macros@III@out\macros@XVII\else\def\macros@III@out{??}\fi\fi\fi\fi\fi\macros@III@out}\newcommand\macros@IV[1][all]{\ifnum\pdfstrcmp{#1}{all}=0\def\macros@IV@out{\{"PLPeak": \{"m\_1percentile": \{"median": "6.32", "error plus": "0.64", "error minus": "1.08", "5th percentile": "5.24", "95th percentile": "6.90"\}, "m\_99percentile": \{"median": "44.77", "error plus": "9.94", "error minus": "6.18", "5th percentile": "38.59", "95th percentile": "55.46"\}\}, "MSpline": \{"m\_1percentile": \{"median": "6.01", "error plus": "0.51", "error minus": "0.19", "5th percentile": "5.82", "95th percentile": "6.59"\}, "m\_99percentile": \{"median": "80.10", "error plus": "11.37", "error minus": "12.53", "5th percentile": "67.57", "95th percentile": "91.26"\}\}, "PLSpline": \{"m\_1percentile": \{"median": "5.83", "error plus": "1.03", "error minus": "1.47", "5th percentile": "4.35", "95th percentile": "6.81"\}, "m\_99percentile": \{"median": "43.20", "error plus": "10.47", "error minus": "4.71", "5th percentile": "38.49", "95th percentile": "54.97"\}\}\}}\else\ifnum\pdfstrcmp{#1}{PLPeak}=0\let\macros@IV@out\macros@XVIII\else\ifnum\pdfstrcmp{#1}{MSpline}=0\let\macros@IV@out\macros@XIX\else\ifnum\pdfstrcmp{#1}{PLSpline}=0\let\macros@IV@out\macros@XX\else\def\macros@IV@out{??}\fi\fi\fi\fi\macros@IV@out}\newcommand\macros@V[1][all]{\ifnum\pdfstrcmp{#1}{all}=0\def\macros@V@out{\{"median": "0.19", "error plus": "0.73", "error minus": "0.73", "5th percentile": "{-}0.54", "95th percentile": "0.89"\}}\else\ifnum\pdfstrcmp{#1}{median}=0\def\macros@V@out{0.19}\else\ifnum\pdfstrcmp{#1}{error plus}=0\def\macros@V@out{0.73}\else\ifnum\pdfstrcmp{#1}{error minus}=0\def\macros@V@out{0.73}\else\ifnum\pdfstrcmp{#1}{5th percentile}=0\def\macros@V@out{{-}0.54}\else\ifnum\pdfstrcmp{#1}{95th percentile}=0\def\macros@V@out{0.89}\else\def\macros@V@out{??}\fi\fi\fi\fi\fi\fi\macros@V@out}\newcommand\macros@VI[1][all]{\ifnum\pdfstrcmp{#1}{all}=0\def\macros@VI@out{\{"median": "0.40", "error plus": "0.64", "error minus": "1.06", "5th percentile": "{-}0.66", "95th percentile": "1.00"\}}\else\ifnum\pdfstrcmp{#1}{median}=0\def\macros@VI@out{0.40}\else\ifnum\pdfstrcmp{#1}{error plus}=0\def\macros@VI@out{0.64}\else\ifnum\pdfstrcmp{#1}{error minus}=0\def\macros@VI@out{1.06}\else\ifnum\pdfstrcmp{#1}{5th percentile}=0\def\macros@VI@out{{-}0.66}\else\ifnum\pdfstrcmp{#1}{95th percentile}=0\def\macros@VI@out{1.00}\else\def\macros@VI@out{??}\fi\fi\fi\fi\fi\fi\macros@VI@out}\newcommand\macros@VII[1][all]{\ifnum\pdfstrcmp{#1}{all}=0\def\macros@VII@out{\{"median": "0.13", "error plus": "0.71", "error minus": "0.70", "5th percentile": "{-}0.57", "95th percentile": "0.84"\}}\else\ifnum\pdfstrcmp{#1}{median}=0\def\macros@VII@out{0.13}\else\ifnum\pdfstrcmp{#1}{error plus}=0\def\macros@VII@out{0.71}\else\ifnum\pdfstrcmp{#1}{error minus}=0\def\macros@VII@out{0.70}\else\ifnum\pdfstrcmp{#1}{5th percentile}=0\def\macros@VII@out{{-}0.57}\else\ifnum\pdfstrcmp{#1}{95th percentile}=0\def\macros@VII@out{0.84}\else\def\macros@VII@out{??}\fi\fi\fi\fi\fi\fi\macros@VII@out}\newcommand\macros@VIII[1][all]{\ifnum\pdfstrcmp{#1}{all}=0\def\macros@VIII@out{\{"median": "0.28", "error plus": "0.72", "error minus": "0.74", "5th percentile": "{-}0.46", "95th percentile": "0.99"\}}\else\ifnum\pdfstrcmp{#1}{median}=0\def\macros@VIII@out{0.28}\else\ifnum\pdfstrcmp{#1}{error plus}=0\def\macros@VIII@out{0.72}\else\ifnum\pdfstrcmp{#1}{error minus}=0\def\macros@VIII@out{0.74}\else\ifnum\pdfstrcmp{#1}{5th percentile}=0\def\macros@VIII@out{{-}0.46}\else\ifnum\pdfstrcmp{#1}{95th percentile}=0\def\macros@VIII@out{0.99}\else\def\macros@VIII@out{??}\fi\fi\fi\fi\fi\fi\macros@VIII@out}\newcommand\macros@IX[1][all]{\ifnum\pdfstrcmp{#1}{all}=0\def\macros@IX@out{\{"median": "0.43", "error plus": "0.19", "error minus": "0.18", "5th percentile": "0.25", "95th percentile": "0.62"\}}\else\ifnum\pdfstrcmp{#1}{median}=0\def\macros@IX@out{0.43}\else\ifnum\pdfstrcmp{#1}{error plus}=0\def\macros@IX@out{0.19}\else\ifnum\pdfstrcmp{#1}{error minus}=0\def\macros@IX@out{0.18}\else\ifnum\pdfstrcmp{#1}{5th percentile}=0\def\macros@IX@out{0.25}\else\ifnum\pdfstrcmp{#1}{95th percentile}=0\def\macros@IX@out{0.62}\else\def\macros@IX@out{??}\fi\fi\fi\fi\fi\fi\macros@IX@out}\newcommand\macros@X[1][all]{\ifnum\pdfstrcmp{#1}{all}=0\def\macros@X@out{\{"median": "0.37", "error plus": "0.19", "error minus": "0.16", "5th percentile": "0.21", "95th percentile": "0.57"\}}\else\ifnum\pdfstrcmp{#1}{median}=0\def\macros@X@out{0.37}\else\ifnum\pdfstrcmp{#1}{error plus}=0\def\macros@X@out{0.19}\else\ifnum\pdfstrcmp{#1}{error minus}=0\def\macros@X@out{0.16}\else\ifnum\pdfstrcmp{#1}{5th percentile}=0\def\macros@X@out{0.21}\else\ifnum\pdfstrcmp{#1}{95th percentile}=0\def\macros@X@out{0.57}\else\def\macros@X@out{??}\fi\fi\fi\fi\fi\fi\macros@X@out}\newcommand\macros@XI[1][all]{\ifnum\pdfstrcmp{#1}{all}=0\def\macros@XI@out{\{"median": "0.37", "error plus": "0.64", "error minus": "0.50", "5th percentile": "{-}0.14", "95th percentile": "1.00"\}}\else\ifnum\pdfstrcmp{#1}{median}=0\def\macros@XI@out{0.37}\else\ifnum\pdfstrcmp{#1}{error plus}=0\def\macros@XI@out{0.64}\else\ifnum\pdfstrcmp{#1}{error minus}=0\def\macros@XI@out{0.50}\else\ifnum\pdfstrcmp{#1}{5th percentile}=0\def\macros@XI@out{{-}0.14}\else\ifnum\pdfstrcmp{#1}{95th percentile}=0\def\macros@XI@out{1.00}\else\def\macros@XI@out{??}\fi\fi\fi\fi\fi\fi\macros@XI@out}\newcommand\macros@XII[1][all]{\ifnum\pdfstrcmp{#1}{all}=0\def\macros@XII@out{\{"median": "0.31", "error plus": "0.66", "error minus": "0.66", "5th percentile": "{-}0.35", "95th percentile": "0.96"\}}\else\ifnum\pdfstrcmp{#1}{median}=0\def\macros@XII@out{0.31}\else\ifnum\pdfstrcmp{#1}{error plus}=0\def\macros@XII@out{0.66}\else\ifnum\pdfstrcmp{#1}{error minus}=0\def\macros@XII@out{0.66}\else\ifnum\pdfstrcmp{#1}{5th percentile}=0\def\macros@XII@out{{-}0.35}\else\ifnum\pdfstrcmp{#1}{95th percentile}=0\def\macros@XII@out{0.96}\else\def\macros@XII@out{??}\fi\fi\fi\fi\fi\fi\macros@XII@out}\newcommand\macros@XIII[1][all]{\ifnum\pdfstrcmp{#1}{all}=0\def\macros@XIII@out{\{"median": "0.34", "error plus": "0.12", "error minus": "0.12", "5th percentile": "0.23", "95th percentile": "0.47"\}}\else\ifnum\pdfstrcmp{#1}{median}=0\def\macros@XIII@out{0.34}\else\ifnum\pdfstrcmp{#1}{error plus}=0\def\macros@XIII@out{0.12}\else\ifnum\pdfstrcmp{#1}{error minus}=0\def\macros@XIII@out{0.12}\else\ifnum\pdfstrcmp{#1}{5th percentile}=0\def\macros@XIII@out{0.23}\else\ifnum\pdfstrcmp{#1}{95th percentile}=0\def\macros@XIII@out{0.47}\else\def\macros@XIII@out{??}\fi\fi\fi\fi\fi\fi\macros@XIII@out}\newcommand\macros@XIV[1][all]{\ifnum\pdfstrcmp{#1}{all}=0\def\macros@XIV@out{\{"FracBelowNeg0p3": \{"median": "0.01", "error plus": "0.02", "error minus": "0.01", "5th percentile": "0.00", "95th percentile": "0.03"\}, "FracBelow0": \{"median": "0.43", "error plus": "0.08", "error minus": "0.15", "5th percentile": "0.28", "95th percentile": "0.49"\}, "PeakChiEff": \{"median": "0.02", "error plus": "0.03", "error minus": "0.02", "5th percentile": "{-}0.01", "95th percentile": "0.05"\}\}}\else\ifnum\pdfstrcmp{#1}{FracBelowNeg0p3}=0\let\macros@XIV@out\macros@XXI\else\ifnum\pdfstrcmp{#1}{FracBelow0}=0\let\macros@XIV@out\macros@XXII\else\ifnum\pdfstrcmp{#1}{PeakChiEff}=0\let\macros@XIV@out\macros@XXIII\else\def\macros@XIV@out{??}\fi\fi\fi\fi\macros@XIV@out}\newcommand\macros@XV[1][all]{\ifnum\pdfstrcmp{#1}{all}=0\def\macros@XV@out{\{"FracBelowNeg0p3": \{"median": "0.02", "error plus": "0.03", "error minus": "0.02", "5th percentile": "0.00", "95th percentile": "0.05"\}, "FracBelow0": \{"median": "0.35", "error plus": "0.10", "error minus": "0.12", "5th percentile": "0.23", "95th percentile": "0.44"\}, "PeakChiEff": \{"median": "0.03", "error plus": "0.04", "error minus": "0.02", "5th percentile": "0.01", "95th percentile": "0.07"\}\}}\else\ifnum\pdfstrcmp{#1}{FracBelowNeg0p3}=0\let\macros@XV@out\macros@XXIV\else\ifnum\pdfstrcmp{#1}{FracBelow0}=0\let\macros@XV@out\macros@XXV\else\ifnum\pdfstrcmp{#1}{PeakChiEff}=0\let\macros@XV@out\macros@XXVI\else\def\macros@XV@out{??}\fi\fi\fi\fi\macros@XV@out}\newcommand\macros@XVI[1][all]{\ifnum\pdfstrcmp{#1}{all}=0\def\macros@XVI@out{\{"FracBelowNeg0p3": \{"median": "0.03", "error plus": "0.02", "error minus": "0.01", "5th percentile": "0.02", "95th percentile": "0.05"\}, "FracBelow0": \{"median": "0.39", "error plus": "0.08", "error minus": "0.07", "5th percentile": "0.32", "95th percentile": "0.47"\}, "PeakChiEff": \{"median": "0.03", "error plus": "0.03", "error minus": "0.03", "5th percentile": "0.00", "95th percentile": "0.05"\}\}}\else\ifnum\pdfstrcmp{#1}{FracBelowNeg0p3}=0\let\macros@XVI@out\macros@XXVII\else\ifnum\pdfstrcmp{#1}{FracBelow0}=0\let\macros@XVI@out\macros@XXVIII\else\ifnum\pdfstrcmp{#1}{PeakChiEff}=0\let\macros@XVI@out\macros@XXIX\else\def\macros@XVI@out{??}\fi\fi\fi\fi\macros@XVI@out}\newcommand\macros@XVII[1][all]{\ifnum\pdfstrcmp{#1}{all}=0\def\macros@XVII@out{\{"FracBelowNeg0p3": \{"median": "0.07", "error plus": "0.05", "error minus": "0.03", "5th percentile": "0.04", "95th percentile": "0.12"\}, "FracBelow0": \{"median": "0.40", "error plus": "0.10", "error minus": "0.10", "5th percentile": "0.29", "95th percentile": "0.50"\}, "PeakChiEff": \{"median": "0.04", "error plus": "0.04", "error minus": "0.04", "5th percentile": "{-}0.00", "95th percentile": "0.07"\}\}}\else\ifnum\pdfstrcmp{#1}{FracBelowNeg0p3}=0\let\macros@XVII@out\macros@XXX\else\ifnum\pdfstrcmp{#1}{FracBelow0}=0\let\macros@XVII@out\macros@XXXI\else\ifnum\pdfstrcmp{#1}{PeakChiEff}=0\let\macros@XVII@out\macros@XXXII\else\def\macros@XVII@out{??}\fi\fi\fi\fi\macros@XVII@out}\newcommand\macros@XVIII[1][all]{\ifnum\pdfstrcmp{#1}{all}=0\def\macros@XVIII@out{\{"m\_1percentile": \{"median": "6.32", "error plus": "0.64", "error minus": "1.08", "5th percentile": "5.24", "95th percentile": "6.90"\}, "m\_99percentile": \{"median": "44.77", "error plus": "9.94", "error minus": "6.18", "5th percentile": "38.59", "95th percentile": "55.46"\}\}}\else\ifnum\pdfstrcmp{#1}{m_1percentile}=0\let\macros@XVIII@out\macros@XXXIII\else\ifnum\pdfstrcmp{#1}{m_99percentile}=0\let\macros@XVIII@out\macros@XXXIV\else\def\macros@XVIII@out{??}\fi\fi\fi\macros@XVIII@out}\newcommand\macros@XIX[1][all]{\ifnum\pdfstrcmp{#1}{all}=0\def\macros@XIX@out{\{"m\_1percentile": \{"median": "6.01", "error plus": "0.51", "error minus": "0.19", "5th percentile": "5.82", "95th percentile": "6.59"\}, "m\_99percentile": \{"median": "80.10", "error plus": "11.37", "error minus": "12.53", "5th percentile": "67.57", "95th percentile": "91.26"\}\}}\else\ifnum\pdfstrcmp{#1}{m_1percentile}=0\let\macros@XIX@out\macros@XXXV\else\ifnum\pdfstrcmp{#1}{m_99percentile}=0\let\macros@XIX@out\macros@XXXVI\else\def\macros@XIX@out{??}\fi\fi\fi\macros@XIX@out}\newcommand\macros@XX[1][all]{\ifnum\pdfstrcmp{#1}{all}=0\def\macros@XX@out{\{"m\_1percentile": \{"median": "5.83", "error plus": "1.03", "error minus": "1.47", "5th percentile": "4.35", "95th percentile": "6.81"\}, "m\_99percentile": \{"median": "43.20", "error plus": "10.47", "error minus": "4.71", "5th percentile": "38.49", "95th percentile": "54.97"\}\}}\else\ifnum\pdfstrcmp{#1}{m_1percentile}=0\let\macros@XX@out\macros@XXXVII\else\ifnum\pdfstrcmp{#1}{m_99percentile}=0\let\macros@XX@out\macros@XXXVIII\else\def\macros@XX@out{??}\fi\fi\fi\macros@XX@out}\newcommand\macros@XXI[1][all]{\ifnum\pdfstrcmp{#1}{all}=0\def\macros@XXI@out{\{"median": "0.01", "error plus": "0.02", "error minus": "0.01", "5th percentile": "0.00", "95th percentile": "0.03"\}}\else\ifnum\pdfstrcmp{#1}{median}=0\def\macros@XXI@out{0.01}\else\ifnum\pdfstrcmp{#1}{error plus}=0\def\macros@XXI@out{0.02}\else\ifnum\pdfstrcmp{#1}{error minus}=0\def\macros@XXI@out{0.01}\else\ifnum\pdfstrcmp{#1}{5th percentile}=0\def\macros@XXI@out{0.00}\else\ifnum\pdfstrcmp{#1}{95th percentile}=0\def\macros@XXI@out{0.03}\else\def\macros@XXI@out{??}\fi\fi\fi\fi\fi\fi\macros@XXI@out}\newcommand\macros@XXII[1][all]{\ifnum\pdfstrcmp{#1}{all}=0\def\macros@XXII@out{\{"median": "0.43", "error plus": "0.08", "error minus": "0.15", "5th percentile": "0.28", "95th percentile": "0.49"\}}\else\ifnum\pdfstrcmp{#1}{median}=0\def\macros@XXII@out{0.43}\else\ifnum\pdfstrcmp{#1}{error plus}=0\def\macros@XXII@out{0.08}\else\ifnum\pdfstrcmp{#1}{error minus}=0\def\macros@XXII@out{0.15}\else\ifnum\pdfstrcmp{#1}{5th percentile}=0\def\macros@XXII@out{0.28}\else\ifnum\pdfstrcmp{#1}{95th percentile}=0\def\macros@XXII@out{0.49}\else\def\macros@XXII@out{??}\fi\fi\fi\fi\fi\fi\macros@XXII@out}\newcommand\macros@XXIII[1][all]{\ifnum\pdfstrcmp{#1}{all}=0\def\macros@XXIII@out{\{"median": "0.02", "error plus": "0.03", "error minus": "0.02", "5th percentile": "{-}0.01", "95th percentile": "0.05"\}}\else\ifnum\pdfstrcmp{#1}{median}=0\def\macros@XXIII@out{0.02}\else\ifnum\pdfstrcmp{#1}{error plus}=0\def\macros@XXIII@out{0.03}\else\ifnum\pdfstrcmp{#1}{error minus}=0\def\macros@XXIII@out{0.02}\else\ifnum\pdfstrcmp{#1}{5th percentile}=0\def\macros@XXIII@out{{-}0.01}\else\ifnum\pdfstrcmp{#1}{95th percentile}=0\def\macros@XXIII@out{0.05}\else\def\macros@XXIII@out{??}\fi\fi\fi\fi\fi\fi\macros@XXIII@out}\newcommand\macros@XXIV[1][all]{\ifnum\pdfstrcmp{#1}{all}=0\def\macros@XXIV@out{\{"median": "0.02", "error plus": "0.03", "error minus": "0.02", "5th percentile": "0.00", "95th percentile": "0.05"\}}\else\ifnum\pdfstrcmp{#1}{median}=0\def\macros@XXIV@out{0.02}\else\ifnum\pdfstrcmp{#1}{error plus}=0\def\macros@XXIV@out{0.03}\else\ifnum\pdfstrcmp{#1}{error minus}=0\def\macros@XXIV@out{0.02}\else\ifnum\pdfstrcmp{#1}{5th percentile}=0\def\macros@XXIV@out{0.00}\else\ifnum\pdfstrcmp{#1}{95th percentile}=0\def\macros@XXIV@out{0.05}\else\def\macros@XXIV@out{??}\fi\fi\fi\fi\fi\fi\macros@XXIV@out}\newcommand\macros@XXV[1][all]{\ifnum\pdfstrcmp{#1}{all}=0\def\macros@XXV@out{\{"median": "0.35", "error plus": "0.10", "error minus": "0.12", "5th percentile": "0.23", "95th percentile": "0.44"\}}\else\ifnum\pdfstrcmp{#1}{median}=0\def\macros@XXV@out{0.35}\else\ifnum\pdfstrcmp{#1}{error plus}=0\def\macros@XXV@out{0.10}\else\ifnum\pdfstrcmp{#1}{error minus}=0\def\macros@XXV@out{0.12}\else\ifnum\pdfstrcmp{#1}{5th percentile}=0\def\macros@XXV@out{0.23}\else\ifnum\pdfstrcmp{#1}{95th percentile}=0\def\macros@XXV@out{0.44}\else\def\macros@XXV@out{??}\fi\fi\fi\fi\fi\fi\macros@XXV@out}\newcommand\macros@XXVI[1][all]{\ifnum\pdfstrcmp{#1}{all}=0\def\macros@XXVI@out{\{"median": "0.03", "error plus": "0.04", "error minus": "0.02", "5th percentile": "0.01", "95th percentile": "0.07"\}}\else\ifnum\pdfstrcmp{#1}{median}=0\def\macros@XXVI@out{0.03}\else\ifnum\pdfstrcmp{#1}{error plus}=0\def\macros@XXVI@out{0.04}\else\ifnum\pdfstrcmp{#1}{error minus}=0\def\macros@XXVI@out{0.02}\else\ifnum\pdfstrcmp{#1}{5th percentile}=0\def\macros@XXVI@out{0.01}\else\ifnum\pdfstrcmp{#1}{95th percentile}=0\def\macros@XXVI@out{0.07}\else\def\macros@XXVI@out{??}\fi\fi\fi\fi\fi\fi\macros@XXVI@out}\newcommand\macros@XXVII[1][all]{\ifnum\pdfstrcmp{#1}{all}=0\def\macros@XXVII@out{\{"median": "0.03", "error plus": "0.02", "error minus": "0.01", "5th percentile": "0.02", "95th percentile": "0.05"\}}\else\ifnum\pdfstrcmp{#1}{median}=0\def\macros@XXVII@out{0.03}\else\ifnum\pdfstrcmp{#1}{error plus}=0\def\macros@XXVII@out{0.02}\else\ifnum\pdfstrcmp{#1}{error minus}=0\def\macros@XXVII@out{0.01}\else\ifnum\pdfstrcmp{#1}{5th percentile}=0\def\macros@XXVII@out{0.02}\else\ifnum\pdfstrcmp{#1}{95th percentile}=0\def\macros@XXVII@out{0.05}\else\def\macros@XXVII@out{??}\fi\fi\fi\fi\fi\fi\macros@XXVII@out}\newcommand\macros@XXVIII[1][all]{\ifnum\pdfstrcmp{#1}{all}=0\def\macros@XXVIII@out{\{"median": "0.39", "error plus": "0.08", "error minus": "0.07", "5th percentile": "0.32", "95th percentile": "0.47"\}}\else\ifnum\pdfstrcmp{#1}{median}=0\def\macros@XXVIII@out{0.39}\else\ifnum\pdfstrcmp{#1}{error plus}=0\def\macros@XXVIII@out{0.08}\else\ifnum\pdfstrcmp{#1}{error minus}=0\def\macros@XXVIII@out{0.07}\else\ifnum\pdfstrcmp{#1}{5th percentile}=0\def\macros@XXVIII@out{0.32}\else\ifnum\pdfstrcmp{#1}{95th percentile}=0\def\macros@XXVIII@out{0.47}\else\def\macros@XXVIII@out{??}\fi\fi\fi\fi\fi\fi\macros@XXVIII@out}\newcommand\macros@XXIX[1][all]{\ifnum\pdfstrcmp{#1}{all}=0\def\macros@XXIX@out{\{"median": "0.03", "error plus": "0.03", "error minus": "0.03", "5th percentile": "0.00", "95th percentile": "0.05"\}}\else\ifnum\pdfstrcmp{#1}{median}=0\def\macros@XXIX@out{0.03}\else\ifnum\pdfstrcmp{#1}{error plus}=0\def\macros@XXIX@out{0.03}\else\ifnum\pdfstrcmp{#1}{error minus}=0\def\macros@XXIX@out{0.03}\else\ifnum\pdfstrcmp{#1}{5th percentile}=0\def\macros@XXIX@out{0.00}\else\ifnum\pdfstrcmp{#1}{95th percentile}=0\def\macros@XXIX@out{0.05}\else\def\macros@XXIX@out{??}\fi\fi\fi\fi\fi\fi\macros@XXIX@out}\newcommand\macros@XXX[1][all]{\ifnum\pdfstrcmp{#1}{all}=0\def\macros@XXX@out{\{"median": "0.07", "error plus": "0.05", "error minus": "0.03", "5th percentile": "0.04", "95th percentile": "0.12"\}}\else\ifnum\pdfstrcmp{#1}{median}=0\def\macros@XXX@out{0.07}\else\ifnum\pdfstrcmp{#1}{error plus}=0\def\macros@XXX@out{0.05}\else\ifnum\pdfstrcmp{#1}{error minus}=0\def\macros@XXX@out{0.03}\else\ifnum\pdfstrcmp{#1}{5th percentile}=0\def\macros@XXX@out{0.04}\else\ifnum\pdfstrcmp{#1}{95th percentile}=0\def\macros@XXX@out{0.12}\else\def\macros@XXX@out{??}\fi\fi\fi\fi\fi\fi\macros@XXX@out}\newcommand\macros@XXXI[1][all]{\ifnum\pdfstrcmp{#1}{all}=0\def\macros@XXXI@out{\{"median": "0.40", "error plus": "0.10", "error minus": "0.10", "5th percentile": "0.29", "95th percentile": "0.50"\}}\else\ifnum\pdfstrcmp{#1}{median}=0\def\macros@XXXI@out{0.40}\else\ifnum\pdfstrcmp{#1}{error plus}=0\def\macros@XXXI@out{0.10}\else\ifnum\pdfstrcmp{#1}{error minus}=0\def\macros@XXXI@out{0.10}\else\ifnum\pdfstrcmp{#1}{5th percentile}=0\def\macros@XXXI@out{0.29}\else\ifnum\pdfstrcmp{#1}{95th percentile}=0\def\macros@XXXI@out{0.50}\else\def\macros@XXXI@out{??}\fi\fi\fi\fi\fi\fi\macros@XXXI@out}\newcommand\macros@XXXII[1][all]{\ifnum\pdfstrcmp{#1}{all}=0\def\macros@XXXII@out{\{"median": "0.04", "error plus": "0.04", "error minus": "0.04", "5th percentile": "{-}0.00", "95th percentile": "0.07"\}}\else\ifnum\pdfstrcmp{#1}{median}=0\def\macros@XXXII@out{0.04}\else\ifnum\pdfstrcmp{#1}{error plus}=0\def\macros@XXXII@out{0.04}\else\ifnum\pdfstrcmp{#1}{error minus}=0\def\macros@XXXII@out{0.04}\else\ifnum\pdfstrcmp{#1}{5th percentile}=0\def\macros@XXXII@out{{-}0.00}\else\ifnum\pdfstrcmp{#1}{95th percentile}=0\def\macros@XXXII@out{0.07}\else\def\macros@XXXII@out{??}\fi\fi\fi\fi\fi\fi\macros@XXXII@out}\newcommand\macros@XXXIII[1][all]{\ifnum\pdfstrcmp{#1}{all}=0\def\macros@XXXIII@out{\{"median": "6.32", "error plus": "0.64", "error minus": "1.08", "5th percentile": "5.24", "95th percentile": "6.90"\}}\else\ifnum\pdfstrcmp{#1}{median}=0\def\macros@XXXIII@out{6.32}\else\ifnum\pdfstrcmp{#1}{error plus}=0\def\macros@XXXIII@out{0.64}\else\ifnum\pdfstrcmp{#1}{error minus}=0\def\macros@XXXIII@out{1.08}\else\ifnum\pdfstrcmp{#1}{5th percentile}=0\def\macros@XXXIII@out{5.24}\else\ifnum\pdfstrcmp{#1}{95th percentile}=0\def\macros@XXXIII@out{6.90}\else\def\macros@XXXIII@out{??}\fi\fi\fi\fi\fi\fi\macros@XXXIII@out}\newcommand\macros@XXXIV[1][all]{\ifnum\pdfstrcmp{#1}{all}=0\def\macros@XXXIV@out{\{"median": "44.77", "error plus": "9.94", "error minus": "6.18", "5th percentile": "38.59", "95th percentile": "55.46"\}}\else\ifnum\pdfstrcmp{#1}{median}=0\def\macros@XXXIV@out{44.77}\else\ifnum\pdfstrcmp{#1}{error plus}=0\def\macros@XXXIV@out{9.94}\else\ifnum\pdfstrcmp{#1}{error minus}=0\def\macros@XXXIV@out{6.18}\else\ifnum\pdfstrcmp{#1}{5th percentile}=0\def\macros@XXXIV@out{38.59}\else\ifnum\pdfstrcmp{#1}{95th percentile}=0\def\macros@XXXIV@out{55.46}\else\def\macros@XXXIV@out{??}\fi\fi\fi\fi\fi\fi\macros@XXXIV@out}\newcommand\macros@XXXV[1][all]{\ifnum\pdfstrcmp{#1}{all}=0\def\macros@XXXV@out{\{"median": "6.01", "error plus": "0.51", "error minus": "0.19", "5th percentile": "5.82", "95th percentile": "6.59"\}}\else\ifnum\pdfstrcmp{#1}{median}=0\def\macros@XXXV@out{6.01}\else\ifnum\pdfstrcmp{#1}{error plus}=0\def\macros@XXXV@out{0.51}\else\ifnum\pdfstrcmp{#1}{error minus}=0\def\macros@XXXV@out{0.19}\else\ifnum\pdfstrcmp{#1}{5th percentile}=0\def\macros@XXXV@out{5.82}\else\ifnum\pdfstrcmp{#1}{95th percentile}=0\def\macros@XXXV@out{6.59}\else\def\macros@XXXV@out{??}\fi\fi\fi\fi\fi\fi\macros@XXXV@out}\newcommand\macros@XXXVI[1][all]{\ifnum\pdfstrcmp{#1}{all}=0\def\macros@XXXVI@out{\{"median": "80.10", "error plus": "11.37", "error minus": "12.53", "5th percentile": "67.57", "95th percentile": "91.26"\}}\else\ifnum\pdfstrcmp{#1}{median}=0\def\macros@XXXVI@out{80.10}\else\ifnum\pdfstrcmp{#1}{error plus}=0\def\macros@XXXVI@out{11.37}\else\ifnum\pdfstrcmp{#1}{error minus}=0\def\macros@XXXVI@out{12.53}\else\ifnum\pdfstrcmp{#1}{5th percentile}=0\def\macros@XXXVI@out{67.57}\else\ifnum\pdfstrcmp{#1}{95th percentile}=0\def\macros@XXXVI@out{91.26}\else\def\macros@XXXVI@out{??}\fi\fi\fi\fi\fi\fi\macros@XXXVI@out}\newcommand\macros@XXXVII[1][all]{\ifnum\pdfstrcmp{#1}{all}=0\def\macros@XXXVII@out{\{"median": "5.83", "error plus": "1.03", "error minus": "1.47", "5th percentile": "4.35", "95th percentile": "6.81"\}}\else\ifnum\pdfstrcmp{#1}{median}=0\def\macros@XXXVII@out{5.83}\else\ifnum\pdfstrcmp{#1}{error plus}=0\def\macros@XXXVII@out{1.03}\else\ifnum\pdfstrcmp{#1}{error minus}=0\def\macros@XXXVII@out{1.47}\else\ifnum\pdfstrcmp{#1}{5th percentile}=0\def\macros@XXXVII@out{4.35}\else\ifnum\pdfstrcmp{#1}{95th percentile}=0\def\macros@XXXVII@out{6.81}\else\def\macros@XXXVII@out{??}\fi\fi\fi\fi\fi\fi\macros@XXXVII@out}\newcommand\macros@XXXVIII[1][all]{\ifnum\pdfstrcmp{#1}{all}=0\def\macros@XXXVIII@out{\{"median": "43.20", "error plus": "10.47", "error minus": "4.71", "5th percentile": "38.49", "95th percentile": "54.97"\}}\else\ifnum\pdfstrcmp{#1}{median}=0\def\macros@XXXVIII@out{43.20}\else\ifnum\pdfstrcmp{#1}{error plus}=0\def\macros@XXXVIII@out{10.47}\else\ifnum\pdfstrcmp{#1}{error minus}=0\def\macros@XXXVIII@out{4.71}\else\ifnum\pdfstrcmp{#1}{5th percentile}=0\def\macros@XXXVIII@out{38.49}\else\ifnum\pdfstrcmp{#1}{95th percentile}=0\def\macros@XXXVIII@out{54.97}\else\def\macros@XXXVIII@out{??}\fi\fi\fi\fi\fi\fi\macros@XXXVIII@out}\makeatother


\begin{document}

\title{Cover Your Basis: Comprehensive Data-Driven Characterization of the Binary Black Hole Population}

\author{Bruce Edelman}
\email{bedelman@uoregon.edu}
\affiliation{Institute  for  Fundamental  Science, Department of Physics, University of Oregon, Eugene, OR 97403, USA}
\author{Ben Farr}
\affiliation{Institute  for  Fundamental  Science, Department of Physics, University of Oregon, Eugene, OR 97403, USA}
\author{Zoheyr Doctor}
\affiliation{Center for Interdisciplinary Exploration and Research in Astrophysics (CIERA), Department of Physics and Astronomy, Northwestern
University, Evanston, IL 60201, USA}


\begin{abstract}                 
We introduce the first complete non-parametric models for the astrophysical distribution of the binary black hole (BBH) population.  Using the 69 confident observations in GWTC-3, we present the most comprehensive data-driven investigation of the BBH population to date, including BBH mass ratio, spin magnitudes and misalignments, and redshift. We report the same features previously recovered 
with similarly flexible models of the mass distribution, most notably the peaks in merger rates at primary masses of ${\sim}10\msun$ and ${\sim}35\msun$.  We infer a distribution for primary spin misalignments that peaks away from alignment, supporting recent work that used a variety of parameterized models to come to the same conclusion. We find broad agreement with the 
previous inferences of the spin magnitude distribution: the majority of BBH spins are small ($a<0.5$), the distribution peaks at $a\sim0.2$, 
and there is mild support for a non-spinning subpopulation, which may be resolved with larger catalogs. 
Using our non-parametric models, we explore whether the data support spin distributions unique for each binary component, and 
find no strong evidence that the distributions differ. We conclude with a discussion of how non-parametric methods in 
gravitational-wave population inference are uniquely poised to complement to the parametric approach as we enter the 
data-rich era of gravitational-wave astronomy.

%When modeling the $\chi_\mathrm{eff}$ distribution with our model we find similar 
%conclusions as previously reported: the distribution is consistent with a non-isotropic distribution (that peaks at $\chi_\mathrm{eff}=\,$\result{$\CIPlusMinus{\macros[ChiEffective][chieff][PeakChiEff]}$}), 
%showing considerable support ($f_{\chi_\mathrm{eff}<0}=\,$\result{$\CIPlusMinus{\macros[ChiEffective][chieff][FracBelow0]}$}) for negative effective spins or misaligned systems.
\end{abstract}

\section{Introduction} \label{sec:intro}

Observations of gravitational waves (GWs) from compact binary mergers are becoming a regular occurrence, 
producing a catalog of events that recently surpassed 90 such detections \citep{GWTC1,gwtc2,GWTC3}. As the catalog continues to grow, so does our understanding of the underlying astrophysical population of compact binaries \citep{o1o2_pop,o3a_pop,o3b_astro_dist}. 
Following numerous improvements to the detectors since the last observing run, the anticipated sensitivities for the upcoming fourth observing run of the LIGO-Virgo-KAGRA (LVK) collaboration suggest detection rates as high as once per \emph{day} \citep{aLIGO,aVIRGO,LVK_prospects,KAGRAProg}. With the formation history of these dense objects encoded in the details of their distribution \citep{Zevin_2017}, the likely doubling in size of the catalog with the next observing run could provide another leap in our understanding of compact binary astrophysics. 
Beyond formation physics, population-level inference of the compact binary catalog has also been shown to provide novel measurements of cosmological parameters \citep{Farr_2019HUB,gwtc3_cosmo,JoseSpectralSirens}, constrain modified gravitational wave propagation \citep{OkounkovaBirefringence,ModGWProp,ModGWProp2}, 
constrain a running Planck mass \citep{Lagos_runningPlanckMass}, search for evidence of ultralight bosons through superradiance \citep{Ng_Boson2021,GWTC2_superradiance_Ng}, 
constrain stellar nuclear reaction rates \citep{Farmer_2019,Farmer_2020}, look for primordial black holes \citep{Ng_2021,KenNgPBH2022}, 
and to constrain physics of neutron stars \citep{Golomb_EOS,LandryRead_NS_Masses2021}. Through a better understanding of the mass, spin, and redshift distributions of 
compact binaries that will come with the increased catalog size, one can probe a wide range of different physical phenomena with even greater fidelity.

The binary black hole (BBH) mass distribution was first found to have structure beyond a smooth power law with simpler parametric models, exhibiting a possible high mass truncation and either a break or a peak at $m_1\sim35-40\,\msun$ \citep{Fishbach_2017,Talbot_2018,o1o2_pop,o3a_pop}. Starting with moderately sized catalog, GWTC-2, more flexible models found signs of additional structure \citep{Tiwari_2021_b,Edelman_2022ApJ}. The evidence supporting these features, 
such as the peak at $m_1\sim10\,\msun$, has only grown after analyzing the latest catalog, GWTC-3, with the same models \citep{o3b_astro_dist,Tiwari_2022ApJ}. 
While this shows the usefulness of data-driven methods with the current relatively small catalog size, they will become more powerful with more observations. The canonical approach to constructing population models has been 
to use simple parametric descriptions (e.g., power laws, beta distributions) that aim to describe the data in the simplest way, employ astrophysically motivated priors where appropriate, then sequentially add 
complexities (e.g., Gaussian peaks) as the data demands. This simple approach was necessary when data was scarce, but as we move into the data-rich catalog era, this approach is already failing to scale.  More flexible and scalable methods, such as non-parametric modeling techniques, will be necessary to continue to extract the full information contained in the compact binary catalog. In contrast to parametric models, flexible and non-parametric models are data-driven and 
contribute little bias to functional form. Non-parametric or semi-parametric methods are particularly useful to search for unexpected features in the 
data, providing meaningful insight of these features that parametric models may fail to capture.

While we eventually hope to uncover hints of binary formation mechanisms in the mass spectrum of BBHs, the distribution of spin properties have been of particular interest.  
The measurement of spin properties of individual binaries often have large uncertainties, but the theorized formation channels are expected to produce distinctly 
different spin distributions \citep{Rodriguez_2016,Gerosa2018, Farr_BinnedSpin, Farr2017Nature, Zevin_2017}. Isolated (or field) formation scenarios predict component spins that are preferentially aligned with the binary's orbital angular momentum, although some small 
misalignment can occur depending on the nature of the supernova kicks as each star collapses to a compact object \citep{Zevin_2022, BaveraBBHSpin, BaveraMassTransfer}. Alternatively, dynamical formation in 
dense environments where many-body interactions between compact objects can result in binary formation and hardening (shrinking of binary orbits) should produce binaries with components' spins distributed isotropically \citep{Rodriguez_2016, Rodriguez_2019}. BBH spins have also been of controversial interest recently, with different parametric approaches to modeling 
the spin distribution coming to different conclusions. Studies have disagreed on the possible existence of a significant zero-spinning subpopulation, as well as the presence of 
significant spin misalignment (i.e. $\cos{\theta_i} < 0.0$) \citep{o3b_astro_dist,RouletGWTC2Pop,BuildBetterSpinModels,GWTC3MonashSpin,Callister_NoEvidence}. 
Another study recently showed that inferences of spin misalignment (or tilts) are sensitive to modeling choices and may not peak at perfectly aligned spins, as is often assumed \citep{spinitasyoulike}. While enlightening,
these recent efforts to improve BBH spin models continue to build sequentially on previous parametric descriptions \citep{BuildBetterSpinModels,Callister_NoEvidence,spinitasyoulike}. 
To ensure we are extracting the full detail the catalog has to offer, we extend our previous non-parametric modeling techniques to include spin magnitudes and tilts, as well as the binary mass ratio and redshift. 
%We present in this manuscript, both to help alleviate the current disputes in the literature and because there has been little work on flexible model development for the 
%spin distribution of BBHs beyond the aligned spin dimension thus far \citep{Tiwari_2021_a,Tiwari_2021_b}.

Polynomial splines have been applied to success across different areas of gravitational-wave astronomy. They have been used to model the gravitational-wave data noise spectrum, 
detector calibration uncertainties, coherent gravitational waveform deviations, and modulations to a power law mass distribution \citep{Littenberg_2015,Edwards_2018,B_Farr_etal_2014,Edelman_2021,Edelman_2022ApJ}
In this paper we highlight how the use of basis-splines can provide a powerful non-parametric modeling approach to the astrophysical distributions of compact 
binaries. We illustrate how one can efficiently model both the mass and spin distributions of merging compact binaries in GWTC-3 with basis splines to infer compact binary population properties using 
hierarchical Bayesian inference. We discuss our results in the context of current literature on compact object populations and how this method complements the simpler lower 
dimensional parametric models in the short run, and will become necessary with future catalogs. Should they appear with more observations, this data-driven approach will provide checks of 
our understanding by uncovering more subtle -- potentially unexpected -- features. The rest of this manuscript is structured as follows: a description of the background of 
basis splines in section \ref{sec:basis_splines}, followed by a presentation of the results of our extensive, data-driven study of the mass and spin distributions of BBHs in GWTC-3 in section 
\ref{sec:results}. We then discuss these results and their astrophysical implications in section \ref{sec:astrodiscussion} and finish with our conclusions in section \ref{sec:conclusion}.
\section{Constructing A Basis} \label{sec:basis_splines}

A common non-parametric method used in many statistical applications is basis splines. A spline function of order $k$, 
is a piece-wise polynomial of order $k$ polynomials stitched together from defined ``knot'' locations across the domain. 
They provide a useful and cheap way to interpolate generically smooth functions from a finite sampling of ``knot'' heights. 
Basis splines of order $k$ are a set of order $k$ polynomials that form a complete basis for any spline function of order $k$. 
Therefore, given an array of knot locations, $\mathbf{t}$ or knot vector, there exists a single unique linear combination of basis splines 
for every possible spline function interpolated from knots, $\mathbf{t}$. The basis components for a given knot vector are constructed  
using the Cox-de Boor recursion formula \citep{deBoor78}. Given knots $t_0$, $t_1$,...,$t_{i+k}$ we start with the base case as:

\begin{figure}[ht!]
    \begin{centering}
        \includegraphics[width=\linewidth]{figures/spline_basis_plot.pdf}
        \caption{Plot showing a ``proper'' (see appendix \ref{sec:psplines}) M-Spline basis of order 3 (cubic) with 20 degrees of freedom and equal weights for each component. 
        In black, we show the resulting spline function with given equal weights and denote the location of the knots with gray x's.}
        \label{fig:spline_basis}
    \end{centering}
    \script{spline_basis_plot.py}
\end{figure}

\begin{equation}
    B_{i,1}(x | \mathbf{t}) = 
    \begin{cases}
        1, & \text{if } t_i \leq x < t_{i+1} \\
        0, & \text{otherwise}
    \end{cases}
\end{equation}

\noindent combined with the recursion relation:

\begin{multline*}
    B_{i,k+1}(x | \mathbf{t}) = \omega_{i,k}(x | \mathbf{t})B_{i,k}(x | \mathbf{t})\\
                                + \big[1-\omega_{i+1,k}(x | \mathbf{t})\big] B_{i+1,k}(x | \mathbf{t})
\end{multline*}

\noindent where above $\omega_{i,k}$ is defined as:

\begin{equation}
\omega_{i,k}(x | \mathbf{t}) =
\begin{cases}
    \frac{x-t_i}{t_{i+k}-t_i}, & t_{i+k} \neq t_i \\
    0, & \text{otherwise}
\end{cases}
\end{equation}

\noindent This is known as the ``B-Spline'' basis after it's inventor de Boor \citep{deBoor78}. The power of basis splines
comes from the fact that one only has to do the somewhat-expensive interpolation once for each set of points at which the spline is evaluated. 
This provides a considerable computational speedup as each evaluation of the spline function becomes a simpler operation: a dot product of a 
matrix and a vector. This straightforward operation is also ideal for optimizations from the use of GPU accelerators, 
enabling expensive Markov chain Monte Carlo (MCMC) based analyses with hundreds of parameters \bruce{citation?}. 
Basis splines can easily be generalized to their two-dimensional analog, producing tensor product basis splines that, 
with this computational advantage, allow for high fidelity modeling of two-dimensional spline functions.

A similar basis, the M-Spline basis \citep{monotone_regression_splines}, has desirable properties when looking to model a probability density function, 
which is the goal of population inference. Each basis component is scaled such that it is normalized over the domain, $\int M_i(x)dx = 1$. This ensures 
that given a vector of basis coefficients, if they sum to unity, the resulting spline function is normalized across the interpolation domain. 
The M-Spline basis differs from B-Splines by the scaling factor that normalizes each component:

\begin{equation}\label{eq:MB_SplineRelation}
M_{i,k} = \frac{k}{t_{i+k} - t_i} B_{i,k}
\end{equation}

\noindent We use this definition of the M-Spline basis for all the basis spline models used in this manuscript. Figure \ref{fig:spline_basis} 
shows an example of a cubic (3rd order) M-Spline basis with 20 degrees of freedom and knots linearly spaced from 0 to 1.
Another important feature of basis splines is that under appropriate prior conditions, one can alleviate sensitivities to arbitrarily 
chosen prior specifications that splines commonly struggle with. We look to the use of penalized splines (or P-Splines) \citep{eilers2021practical,BayesianPSplines,Jullion2007RobustSO}, 
where one adds a smoothing prior based on the difference of basis spline coefficients. This allows one to populate the domain with more dense knots 
without the worry of extra variance in the inferred spline functions. We discuss the details of our smoothing prior implementation 
in more detail in Appendix \ref{sec:psplines}, along with our specific prior and basis choices for each model in Appendix \ref{sec:modelpriors}.

\section{Population Inference of GWTC-3 BBHs} \label{sec:results}

\subsection{Mass Distribution} \label{sec:mass_dist}

We apply our newly constructed models to the BBH analysis that was done in the astrophysical population companion paper to GWTC-3. 
We find consistent features in the mass disribution with what was inferred with the \textsc{PowerlawPeak} and \textsc{PowerlawSpline} 
mass models. In particular our MSpline model finds a peak in merger rate density at both 10 and 35 soloar masses, agreeing with those reported
in \bruce{CITE o3b paper}. We also find a significantly less declining mass distribution that has considerably larger support for higher mass 
mergers when compared to the other power-law based models. Another interesting consequence of this is that we thus infer a much less steeply
evolving merger rate with redshift. It is known that there is a strong correlation between the primary mass power law index, $alpha$, and the 
redshift power-law index, $\lambda_z$. When one infers a steeper mass distribution -- the inferred rate must drastically increase 
with resdshift to explain the few higher mass events in our catalog, but when one infers a flatter distribution like the MSPline inferred
distribition in plot \ref{fig:mass_distribution}, the higher mass mergers are consistent with the local merger rate denstiy as a function of mass, 
and thus the model does not need to have as steep of power-law in redshift. With this more flexible mass population model, we now 
find that the merger rate is fully consistent with being unifom with co-moving volume, rather than the confidently evolving merger rate 
conclusion from \bruce{cite O3b}. 

\begin{figure*}[ht!]
    \script{mass_distribution_plot.py}
    \begin{centering}
        \includegraphics[width=\textwidth]{figures/mass_distribution_plot.pdf}
        \caption{Plot showing the primary mass (left) and mass ratio (right) distributions inferred with the 
        \textsc{MSpline} model (black) with 40 knots for $m_1$, and 15 knots for $q$. We show the results for the \textsc{PowerlawPeak} (blue) and 
        \textsc{PowerlawSpline} (orange) models fromthe LVK's GWTC-3 population analyses. Solid lines show the median of the posterior while the shaded
        bands show the 90\% credible intervals.}
        \label{fig:mass_distribution}
    \end{centering}
\end{figure*}


In addition to the primary mass distirbution there is some mild evidence for a mass ratio distibution that is not well described by a power law, 
which is how the mass ratio distribution is commonly modeled. In particular our smoothed MSpline model finds a decrease in the merger rate at $q\sim1$, 
while agreeing well with the power-law on the decline in rate from $q \in [0.4, 0.8]$, and a much larger tail extending to more asymmetric
mass ratios. 

\subsection{Spin Distributions} \label{sec:spin_dist}

The spins distribution of merging combact binaries provides a very useful insight into the formation history of the binary. In particular the two main subcategories, 
the isolated and dynamical formation channels, would produce distinct signatures in the spin distribution when looking at a population of observations. Isolated formation
is where the only interactions with the binary are between each other. As the binary evolves and mass transfer happens between the two components, we expect the orientation
of each components spin vector to be preferentially aligned with the orbital angular mommentumof the system. Some misalignment is epected to occur due to the kicks from 
each star exploding in supernovae leaving behind the compact objects. Conversely, the dynamical channel     

\begin{figure*}[ht!]
    \script{component_spin_distribution_plot.py}
    \begin{centering}
        \includegraphics[width=\textwidth]{figures/component_spin_distribution_plot.pdf}
        \caption{Plot showing the inferred component spin magnitude (top) and spin tilt (bottom) distributions inferred with the 9-knot \textsc{MSpline} models. 
        There are two independent \textsc{MSpline} (black) models for each primary/secondary spin magnitude and spin tilt. We compare by showing the results for the
        \textsc{Default} (iid) spin model used with the \textsc{PowerlawPeak} mass model (blue) and the \textsc{PowerlawSpline} mass model (orange).}
        \label{fig:component_spin_distribution}
    \end{centering}
\end{figure*}

\begin{figure*}
    \script{iid_component_spin_distribution_plot.py}
    \begin{centering}
        \includegraphics[width=\textwidth]{figures/iid_component_spin_distribution_plot.pdf}
        \caption{Plot showing the inferred component spin magnitude (left) and spin tilt (right) distributions inferred with the 9-knot \textsc{MSpline} models. 
        Each components spin magnitude and tilt are assumed to be IID. We compare by showing the results for the
        \textsc{Default} (iid) spin model used with the \textsc{PowerlawPeak} mass model (blue) and the \textsc{PowerlawSpline} mass model (orange).}
        \label{fig:iid_component_spin_distribution}
    \end{centering}
\end{figure*}
\section{Discussion}\label{sec:astrodiscussion}

\section{Conclusions}\label{sec:conclusion}

\bruce{have subsection in results.tex to discuss the astro interpretation of our results and connect to literature. 
I think main points for conclusion are to sum up conclusions with one last call to literature for most important parts, 
followed by short discussion of the outlook going into O4 and how this method will be especially useful as we rapidly approach
GW's data rich era}

\section{Acknowledgements}\label{sec:acknowledments}
We thank Will Farr, Tom Callister, and Maya Fishbach for useful discussions during the preparation of this manuscript, 
along with helpful comments on an early draft by Tom Callister and Jaxen Godfrey.
This research has made use of data, software and/or web tools obtained from the Gravitational Wave Open Science Center 
(\url{https://www.gw-openscience.org/}), a service of LIGO Laboratory, the LIGO Scientific Collaboration and the Virgo Collaboration. 
The authors are grateful for computational resources provided by the LIGO Laboratory and supported by National Science Foundation Grants PHY-0757058 and PHY-0823459.  
This work benefited from access to the University of Oregon high performance computer, Talapas. This material is based upon work supported 
in part by the National Science Foundation under Grant PHY-1807046 and work supported by NSF's LIGO Laboratory which is a major facility 
fully funded by the National Science Foundation.
\software{
\textsc{Matplotlib}~\citep{Hunter:2007},
\textsc{NumPy}~\citep{harris2020array},
\textsc{SciPy}~\citep{2020SciPy-NMeth}
\textsc{AstroPy}~\citep{2018AJ....156..123A},
\textsc{Jax}~\citep{jax},
\textsc{NumPyro}~\citep{pyro,numpyro},
}
\bibliography{bib}{}
\bibliographystyle{aasjournal}


\appendix
\section{Hierarchical Bayesian Inference} \label{sec:hierarchical_inference}

% This appendix goes over in detail the hierarchical bayeisan inference framework
We use hierarchical Bayesian inference to infer the population properties of CBCs. We want to infer the number density of merging CBCs 
in the universe and how this can change with their masses, spins, etc. Often times it is useful to formulate the question in terms of the 
merger rates which is the number of mergers per $Gpc^{3}$ co-moving volume per year. For a set of hyper-parameters, $\Lambda$, $\lambda$, and local ($z=0$) 
merger rate density, $\mathcal{R}_0$, we write the overall number density of BBH mergers in the universe as: 

\begin{equation} \label{number_density}
     \frac{dN(\theta, z | \mathcal{R}_0, \Lambda, \lambda)}{d\theta dz} = \frac{dV_c}{dz}\bigg(\frac{T_\mathrm{obs}}{1+z}\bigg) \frac{d\mathcal{R}(\theta, z | \mathcal{R}_0, \Lambda, \lambda)}{d\theta} = \mathcal{R}_0 p(\theta | \Lambda) p(z | \lambda)
\end{equation}

\noindent
Where up above, we denote the co-moving volume element as $dV_c$ \citep{hogg_cosmo}, and $T_\mathrm{obs}$ as the observing time period that produced the 
catalog with the related factor of $1+z$ converting this detector-frame time to source-frame. We assume a LambdaCDM cosmology using 
the cosmological parameters from \citet{Planck2015}. We model the merger rate evolving with redshift following a power law distribution: 
$p(z|\lambda) \propto \frac{dV_c}{dz}\frac{1}{1+z}(1+z)^\lambda$. When integrating equation \ref{number_density} across all $\theta$
and out to some maximum redshift, $z_\mathrm{max}$, we get the total number of CBCs in the universe out to that redshift. We follow previous notations, \
letting $\{d_i\}$ represent the set of data from $N_\mathrm{obs}$ CBCs observed with gravitational waves. The merger rate is then described as an inhomogenous 
Poisson process and after imposing the usual log-uniform prior on the merger rate, we marginalize out merger rate and arrive at the posterior
distribution of our hyper-parameters, $\Lambda$ \citep{Mandel_2019, Vitale_2021}.

\begin{equation}
    p\left(\Lambda, \lambda | \{d_i\}\right) \frac{p(\Lambda)p(\lambda)}{\xi(\Lambda,\lambda)^{N_\mathrm{obs}}} \prod_{i=1}^{N_\mathrm{obs}} \bigg[ \frac{1}{K_i} \sum_{j=1}^{K_i} \frac{p(\theta^{i,j}|\Lambda)p(z^{i,j}|\lambda)}{\pi(\theta, z^{i,j})} \bigg]
\end{equation}

\noindent
where, we replaced the integrals over each event's likelihood with ensemble averages over $K_i$ posterior samples. Above, $j$
indexes the $K_i$ posterior samples from each event and $\pi(\theta, z)$ is the default prior used by parameter estimations that 
produced the posterior samples for each event. In the analyses of GWTC-2 the default prior used is uniform in detector frame masses, 
component spins and Euclidean volume. The corresponding prior evaluated in source frame primary mass, mass ratio, component spins 
and redshift is:

\begin{equation}
    \pi(m_1, q, a_1, a_2, \cos{\theta_1}, \cos{\theta_2}, z) \propto \frac{1}{4} m_1 (1+z)^2 D_L^2(z) \frac{dD_L}{dz}
\end{equation}

\noindent where $D_L$ is the luminosity distance. To carefully incorporate selection effects to our model we need to quantify the detection efficiency,
$\xi(\Lambda, \lambda)$, of the search pipelines that were used to create GWTC-3, at a given population distribution described by $\Lambda$ and $\lambda$.
 
\begin{equation}
     \xi(\Lambda, \lambda) = \int d\theta dz P_\mathrm{det}(\theta, z)p(\theta | \Lambda) p(z | \lambda)
\end{equation}
 
\noindent
To estimate this integral we use a software injection campaign where gravitational waveforms from a large population of simulated sources. 
These simulated waevforms are put into real detector data, and then this data is evaluated with the same search pipelines that were used to 
produce the catalog we are analyzing. With these search results in hand, we use importance sampling and evaluate the integral 
with the monte carlo sum estimate $\mu$, and its corresponding variance and effective number of samples:

\begin{equation} \label{xi}
     \xi(\Lambda, \lambda) \approx \mu(\Lambda, \lambda) \frac{1}{N_\mathrm{inj}} \sum_{i=1}^{N_\mathrm{found}} \frac{p(\theta^i | \Lambda) p(z^i | \lambda)}{p_\mathrm{inj}(\theta, z^i)}
\end{equation}

\begin{equation}
    \sigma^2(\Lambda, \lambda) \equiv \frac{\mu^2(\Lambda, \lambda)}{N_\mathrm{eff}} \simeq \frac{1}{N^2_\mathrm{inj}} \sum_{i=1}^{N_\mathrm{found}} \bigg[\frac{p(\theta | \Lambda) p(z | \lambda)}{p_\mathrm{inj}(\theta, z)}\bigg]^2 - \frac{\mu^2(\Lambda, \lambda)}{N_\mathrm{inj}}
\end{equation}

\noindent
Where the sums indexes only over the $N_\mathrm{found}$ injections that were successfully detected out of $N_\mathrm{inj}$ total injections, 
and $p_\mathrm{inj}(\theta, z)$ is the reference distribution from which the injections were drawn. Additionally, we follow the procedure 
outlined in \citet{Farr_2019} to marginalize the uncertainty in our estimate of $\xi(\Lambda, \lambda)$, in which we verify that $N_\mathrm{eff}$ is 
sufficiently high after re-weighting the injections to a given population (i.e. $N_\mathrm{eff} > 4N_\mathrm{obs}$). 
The total hyper-posterior marginalized over the merger rate and the uncertainty in the monte carlo integral calculating $\xi(\Lambda, \lambda)$ \citep{Farr_2019}, as:

\begin{equation}\label{importance-posterior}
    \log p\left(\Lambda, \lambda | \{d_i\}\right) \propto \sum_{i=1}^{N_\mathrm{obs}} \log \bigg[ \frac{1}{K_i} \sum_{j=1}^{K_i} \frac{p(\theta^{i,j}|\Lambda)p(z^{i,j}|\lambda)}{\pi(\theta^{i,j}, z^{i,j})} \bigg] -  \\
    N_\mathrm{obs} \log \mu(\Lambda, \lambda) + \frac{3N_\mathrm{obs} + N_\mathrm{obs}^2}{2N_\mathrm{eff}} + \mathcal{O}(N_\mathrm{eff}^{-2}).
\end{equation}

We explicitly enumerate each of the models used in this work for $p(\theta|\Lambda)$, along with 
their respective hyper-parameters and prior distributions in the next section. To calculate marginal likelihoods and draw 
samples of the hyper parameters from the hierarchical posterior distribution shown in equation \ref{importance-posterior}, we use the 
\textsc{Bilby} \citep{Ashton_2019, bilby_gwtc1} and \textsc{GWPopulation} \citep{Talbot_2019} Bayesian inference software libraries with 
the \textsc{Dynesty} dynamic nested sampling algorithm \citep{Speagle_2020}. We also make use of the hamiltonian monte carlo algorithms in 
\textsc{numpyro} and use \textsc{jax} to calculate our likelihoods \bruce{FIND citations for these}.

\section{Models and Hyper-Parameters} \label{sec:hypparams}

\section{Validation Studies} \label{sec:validation}

\end{document}