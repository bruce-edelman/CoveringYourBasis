\section{Binary Black Hole Population Inference with GWTC-3} \label{sec:results}

We use hierarchical Bayesian inference (see Appendix \ref{sec:hierarchical_inference}) to simultaneosly infer the astrophysical mass, spin and redshift distributions of 
a binary black holes (BBHs) given a catalog of gravitational wave observations. We choose to parameterize the binaries' masses with the primary (defined as the more massive component) mass 
and the mass ratio -- defined as $q=\frac{m_2}{m_1}$, bounded in the range from 0 to 1. We choose to model 4 of the 6 total 
spin degrees of freedom of a binary merger with each component spin magnitude $a_1$ and $a_2$, along with the cosine of the tilt angle of each component, 
$\cos{\theta_1}$ and $\cos{\theta_2}$. The tilt angle is defined as the angle between each components' spin vector and the binaries orbital angular mommentum vector. 
We additionally will fit a population model on the redshift or luminosity distance distribution of BBHs. We assume a $\Lambda\mathrm{CDM}$ cosmology using the 
parameters from the Planck 2015 results \cite{Planck2015}, which defines an analytical mapping between each event's inferred
luminosity distance and it's redshift. While we choose to use MSpline distibutions to model the primary mass and spin distributions, we 
model the mass ratio distribution with a power law, i.e. $p(q | m_1, m_\mathrm{min}, \beta_q) \propto q^{\beta_q} \Theta(qm_1 - m_\mathrm{min}) \Theta(m_1 - qm_1)$. 
Where the $\Theta$ is the Heaviside step function which ensures $m_2$ is within the allowed range, [$m_\mathrm{min}$, $m_1$].
We model the evolution of the merger rate with redshift with the \textsc{PowrerlawRedshift} model \cite{Fishbach_2018redshift}, which is desribed by 
$p(z|\lambda_z)\propto \frac{dV_c}{dz} \frac{T_\mathrm{obs}}{1+z} (1+z)^\lambda_z$. We describe our specific prior choices on the hyper-parameters and each model 
in more detail in Appendix \ref{sec:modelpriors}. We choose the same cut on the recent GWTC-3 catalog as was done in the LVK's accompanying population analysis, \bruce{cite}.
Keeping all possible BBH mergers with false alarm rates less than 1 per year, we have 70 oservations. We choose to emit the currently not well understood event, GW190814 \cite{190814disc},
as it was concluded to be an outlier of the rest of the BBH population in both GWTC-2 and GWTC-3 \cite{o3a_pop}. This leaves us with a catlog of 69 BBH mergers observerd oveer a period of about 2 years, from
which we want to infer the population properties of. We start by looking towards the mass distribution.

\subsection{BBH Mass Distibution} \label{sec:mass_dist}

PLPeak: $m_{1\%} = $\result{$\CIPlusMinus{\macros[MassDistribution][PLPeak][m_1percentile]}$}, $m_{99\%} = $\result{$\CIPlusMinus{\macros[MassDistribution][PLPeak][m_99percentile]}$}


PLSpline: $m_{1\%} = $\result{$\CIPlusMinus{\macros[MassDistribution][PLSpline][m_1percentile]}$}, $m_{99\%} = $\result{$\CIPlusMinus{\macros[MassDistribution][PLSpline][m_99percentile]}$}


MSpline: $m_{1\%} = $\result{$\CIPlusMinus{\macros[MassDistribution][MSpline][m_1percentile]}$}, $m_{99\%} = $\result{$\CIPlusMinus{\macros[MassDistribution][MSpline][m_99percentile]}$}

We apply our newly constructed models to the BBH analysis that was done in the astrophysical population companion paper to GWTC-3. 
We find consistent features in the mass disribution with what was inferred with the \textsc{PowerlawPeak} and \textsc{PowerlawSpline} 
mass models. In particular our MSpline model finds a peak in merger rate density at both 10 and 35 soloar masses, agreeing with those reported
in \bruce{CITE o3b paper}. We also find a significantly less declining mass distribution that has considerably larger support for higher mass 
mergers when compared to the other power-law based models. Another interesting consequence of this is that we thus infer a much less steeply
evolving merger rate with redshift. It is known that there is a strong correlation between the primary mass power law index, $alpha$, and the 
redshift power-law index, $\lambda_z$. When one infers a steeper mass distribution -- the inferred rate must drastically increase 
with resdshift to explain the few higher mass events in our catalog, but when one infers a flatter distribution like the MSPline inferred
distribition in plot \ref{fig:mass_distribution}, the higher mass mergers are consistent with the local merger rate denstiy as a function of mass, 
and thus the model does not need to have as steep of power-law in redshift. With this more flexible mass population model, we now 
find that the merger rate is fully consistent with being unifom with co-moving volume, rather than the confidently evolving merger rate 
conclusion from \bruce{cite O3b}. 

\begin{figure*}[ht!]
    \script{mass_distribution_plot.py}
    \begin{centering}
        \includegraphics[width=\textwidth]{figures/mass_distribution_plot.pdf}
        \caption{Plot showing the primary mass distribution inferred with the \textsc{MSpline} model with 40 knots for $m_1$. We show the results for the 
        \textsc{PowerlawSpline} (blue) model from the LVK's GWTC-3 population analyses.}
        \label{fig:mass_distribution}
    \end{centering}
\end{figure*}


In addition to the primary mass distirbution there is some mild evidence for a mass ratio distibution that is not well described by a power law, 
which is how the mass ratio distribution is commonly modeled. In particular our smoothed MSpline model finds a decrease in the merger rate at $q\sim1$, 
while agreeing well with the power-law on the decline in rate from $q \in [0.4, 0.8]$, and a much larger tail extending to more asymmetric
mass ratios. 

\subsection{BBH Component Spin Distributions} \label{sec:spin_dist}

\subsubsection{Spin Magnitude}

\subsubsection{Spin Orientation}

The spins distribution of merging combact binaries provides a very useful insight into the formation history of the binary. In particular the two main subcategories, 
the isolated and dynamical formation channels, would produce distinct signatures in the spin distribution when looking at a population of observations. Isolated formation
is where the only interactions with the binary are between each other. As the binary evolves and mass transfer happens between the two components, we expect the orientation
of each components spin vector to be preferentially aligned with the orbital angular mommentumof the system. Some misalignment is epected to occur due to the kicks from 
each star exploding in supernovae leaving behind the compact objects. Conversely, the dynamical channel     
\begin{figure}
    \script{plot_iid_spinmag.py}
    \includegraphics[width=0.45\textwidth]{figures/iid_spinmag.pdf}
    \caption{Caption.}
    \label{fig:iid_spinmag_dist}
\end{figure}

\begin{figure}
    \script{plot_iid_spintilt.py}
    \includegraphics[width=0.45\textwidth]{figures/iid_spintilt.pdf}
    \caption{Caption.}
    \label{fig:iid_spintilt_dist}
\end{figure}

Comment here are the inferences of iid model vs Default and then iid vs independent test pointing out possible reasons for why spin tilt 
does not peak at cos theta = 1 i.e. salvo's paper

\subsection{The Effective Spin Dimension}

Transform comp spin results to effective dims to see how strong changes in mag/tilt dont significantly affect our best constrained/measured
effective spins

Present results fitting chi eff dist with MSpline I am making a change. 

\begin{figure}
    \script{plot_ind_spinmag.py}
    \includegraphics[width=0.45\textwidth]{figures/ind_spinmag.pdf}
    \caption{Caption.}
    \label{fig:ind_spinmag_dist}
\end{figure}

\begin{figure}
    \script{plot_ind_spintilt.py}
        \includegraphics[width=0.45\textwidth]{figures/ind_spintilt.pdf}
        \caption{Caption.}
        \label{fig:ind_spintilt_dist}
\end{figure}


\begin{figure}
    \script{plot_chieff.py}
    \includegraphics[width=0.45\textwidth]{figures/chi_eff.pdf}
    \caption{Caption.}
    \label{fig:chieff_dist}
\end{figure}