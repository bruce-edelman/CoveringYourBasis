\section{Binary Black Hole Population Inference with GWTC-3} \label{sec:results}

\begin{figure*}[ht!]
    \begin{centering}
        \includegraphics[width=\linewidth]{figures/mass_distribution_plot.pdf}
        \caption{The primary mass distribution inferred with the M-Spline model (red), with 50 knots spaced linearly in $\log m_1$, from the 
        minimum observed value to 100\msun. The solid line shows the population predictive distribution (PPD) and the shaded region, 
        the 90\% credible interval. We show the inferred PPD from the \textsc{PowerlawPeak} (blue) and \textsc{PowerlawSpline} (green) models 
        from the LVK's GWTC-3 population analyses \citep{o3b_astro_dist}.}
        \label{fig:mass_distribution}
    \end{centering}
    \script{mass_distribution_plot.py}
\end{figure*}

We use hierarchical Bayesian inference (see Appendix \ref{sec:hierarchical_inference}) to simultaneously infer the astrophysical mass, spin and redshift distributions of 
binary black holes (BBHs) given a catalog of gravitational wave observations. We parameterize the binaries' masses with the primary (defined as the more massive component) mass 
and the mass ratio -- defined as $q=\frac{m_2}{m_1}$, bounded in the range from 0 to 1. Furthermore, we model 4 of the 6 total 
spin degrees of freedom of a binary merger with each component spin magnitude $a_1$ and $a_2$, along with the cosine of the tilt angle of each component, 
$\cos{\theta_1}$ and $\cos{\theta_2}$. The tilt angle is defined as the angle between each components' spin vector and the binaries orbital angular momentum vector. 
We assume the polar spin angles are uniformly distributed in the orbital plane. While we choose to use M-Spline distributions to model the primary mass and spin distributions, we 
model the mass ratio distribution with a power law, i.e. $p(q | m_1, m_\mathrm{min}, \beta) \propto q^{\beta} \Theta(qm_1 - m_\mathrm{min}) \Theta(m_1 - qm_1)$, where $\beta$ is 
the power law index and $\Theta$ is the Heaviside step function which ensures $m_2$ is within the allowed range, [$m_\mathrm{min}$, $m_1$] \citep{Talbot_2018,o1o2_pop,o3a_pop}.
Additionally, we fit a population model on the redshift or luminosity distance distribution of BBHs, assuming a $\Lambda\mathrm{CDM}$ cosmology defined by the parameters 
from the Planck 2015 results \citep{Planck2015}. This defines an analytical mapping between each event's inferred luminosity distance, and it's redshift, which we now use interchangeably. 
We model the evolution of the merger rate with redshift with the \textsc{PowerlawRedshift} model, described by $p(z|\lambda_z)\propto \frac{dV_c}{dz}(1+z)^{\lambda_z-1}$ \citep{Fishbach_2018redshift}. 
We describe our specific prior choices on the hyperparameters and each model in more detail in Appendix \ref{sec:modelpriors}. Following the same cut on the recent GWTC-3 catalog done in the LVK's 
accompanying BBH population analysis, we have 70 possible BBH mergers with false alarm rates less than 1 per year \citep{GWTC3,o3b_astro_dist}. Since it was concluded to be an outlier of the rest of 
the BBH population in both GWTC-2 and GWTC-3, we choose to omit the poorly understood event, GW190814 \citep{190814disc,o3a_pop,o3b_astro_dist,Essick_2022}. This leaves us with 
a catalog of 69 confident BBH mergers observed over a period of about 2 years, from which we want to infer its population properties. Our study provides the first comprehensive 
data-driven investigation of the BBH mass and spin distributions, uncovering new insights and corroborating those found with other methods. 
We start by looking towards our model's inferences of the mass distribution.

\subsection{Binary Black Hole Masses} \label{sec:mass_dist}

We apply our newly constructed models to the BBH analysis that was done in the astrophysical population companion paper to GWTC-3. 
Figure \ref{fig:mass_distribution} shows how the M-Spline model (red) found consistent features in the mass distribution with 
what was inferred with the \textsc{PowerlawPeak} and \textsc{PowerlawSpline} mass models \citep{Talbot_2018,o3a_pop,Edelman_2022ApJ,o3b_astro_dist}. 
In particular our M-Spline model finds a peak in merger rate density as a function of primary mass, at both $\sim10\msun$ and $\sim35\msun$ agreeing with those 
reported using the same dataset in \citet{o3b_astro_dist}. We also find a shallow mass distribution that has considerably larger support for higher mass mergers when 
compared to power-law based models. The M-Spline model finds that 75\% percent of primary masses are below \result{$\CIPlusMinus{\macros[MassDistribution][MSpline][m_75percentile]}$}, 
compared to \result{$\CIPlusMinus{\macros[MassDistribution][PLPeak][m_75percentile]}$} and \result{$\CIPlusMinus{\macros[MassDistribution][PLSpline][m_75percentile]}$}, found with 
the \textsc{PowerlawPeak} and \textsc{PowerlawSpline} models respectively. An interesting consequence of this is that we do not find a confidently evolving merger rate with redshift, 
as inferred with power-law based mass models in \citet{o3b_astro_dist}. There is a strong correlation measured between the primary mass power law index, 
$\alpha$, and the redshift power-law index, $\lambda_z$. When one infers a steeper mass distribution, 
the inferred rate must drastically increase with redshift to explain the highest mass events in the catalog. However, when one infers a 
flatter distribution, as the M-Spline does in figure \ref{fig:mass_distribution}, the higher-mass mergers are consistent with 
the local merger rate density as a function of mass given the selection effects. With our more flexible mass model, we find that the merger rate 
is consistent with being uniform with co-moving volume, with $\lambda_z = $\result{$\CIPlusMinus{\macros[MSplineIIDCompSpins][lamb]}$}, compared with 
\citet{o3b_astro_dist}'s reported value of $\lambda_z = $\result{$\CIPlusMinus{\macros[PLPeak][lamb]}$} with the \textsc{PowerlawPeak} mass model.

We find a similar mass ratio distribution to what was found with the \textsc{PowerlawPeak} model in \citet{o3b_astro_dist}. 
With the M-Spline model on primary mass the mass ratio power law index is $\beta = $\result{$\CIPlusMinus{\macros[MSplineIIDCompSpins][beta]}$}, 
compared to $\beta = $\result{$\CIPlusMinus{\macros[PLPeak][beta]}$} for \textsc{PowerlawPeak}. We find that BBH mergers favor more equal mass mergers
as would be expected of both main formation channels due to binary mass transfer in isolated formation \bruce{double check this statement with pop synth lit} 
and three-body interactions that preferentially eject the lightest of the three compact objects, in dynamical formation \bruce{cite these}. 
We also find no evidence for a sharp fall off in merger rate following the pileup at $\sim35\,\msun$, which is expected if such a pileup was due 
to pulsational pair instability supernovae (PPISNe). The lack of any high mass truncation along with the peak at 35\msun, significantly lower 
than expected, may pose challenges for conventional stellar evolution theory and could be a sign of the significant 
presence of subpopulations that may avoid PISN as the binary forms. 

\subsection{Binary Black Hole Spins} \label{sec:spin_dist}

\begin{figure}
    \begin{centering}    
        \includegraphics[width=\linewidth]{figures/iid_spinmag.pdf}
        \caption{The spin magnitude distribution inferred with the M-Spline model (red), with 16 knots spaced linearly from 0 to 1, and 
        assuming the components are IID. The solid line shows the population predictive distribution (PPD) and the shaded region, the 90\% credible interval. 
        We show the inferred PPD from the \textsc{Default} (blue) model from \citet{o3b_astro_dist}, the LVK's GWTC-3 population analyses.}
        \label{fig:iid_spinmag_dist}
    \end{centering}
    \script{plot_iid_spinmag.py}
\end{figure}

\subsubsection{Spin Magnitude}

The \textsc{Default} spin model describes the spin magnitude of both components as identical and independently distributed (IID) non-singular Beta distributions. \bruce{cite Wysocki/Talbot}
The Beta distribution provides a simple 2-parameter model that can produce a wide range of functional forms on the unit interval. However, the constraint that keeps 
the Beta distribution non-singular (i.e. $\alpha>1$ and $\beta>1$) enforces a spin magnitude that always has $p(a_i=0) = 0$. In reality, it is expected that 
all physical bodies must have at least some non-zero angular momentum, studies have proposed the existence of a distinct subpopulation of non-spinning or 
negligibly spinning black holes that can elude discovery with such a model \citep{FullerMa2019,RouletGWTC2Pop,BuildBetterSpinModels,Callister_NoEvidence,GWTC3MonashSpin}. 

\begin{figure} 
    \begin{centering}
        \includegraphics[width=\linewidth]{figures/ind_spinmag.pdf}
        \caption{The primary (orange) and secondary (olive) spin magnitude distributions inferred with the M-Spline model, 
        with 16 knots spaced linearly from 0 to 1. The solid line shows the population predictive distribution (PPD) and the shaded region, the 90\% credible interval. 
        We show the inferred PPD from the \textsc{Default} (blue) model from \citet{o3b_astro_dist}, the LVK's GWTC-3 population analyses.}
        \label{fig:ind_spinmag_dist}
    \end{centering}
    \script{plot_ind_spinmag.py}
\end{figure}

We model the spin magnitude distributions as IID M-Spline distributions as a flexible check
to the assumptions of the \textsc{Default} model. Figure \ref{fig:iid_spinmag_dist} shows the inferred spin magnitude distribution with the M-Spline model, 
compared with the \textsc{Default} model. The M-Spline model finds broad agreement with the beta distribution, both peaking near ~0.2, with 90\% 
of BBH spins below \result{$\CIPlusMinus{\macros[MSplineIIDCompSpins][a_90percentile]}$} at 90\% credibility. The regions of parameter space where they don't agree as well are at 
the boundaries. The M-Spline model has no enforcement for vanishing support at the extremal values like the Beta distribution, 
allowing it to probe the zero-spin question. We find broad support, with large variance, for non-zero probabilities at $a_i=0$, but cannot confidently determine the presence of 
a significant non-spinning subpopulation, corroborating similar recent conclusions \citep{BuildBetterSpinModels,Callister_NoEvidence,GWTC3MonashSpin}. 
We now repeat the same analysis as above but with independent M-Spline distributions for each spin magnitude component. In figure \ref{fig:ind_spinmag_dist} 
we show the inferred primary (orange), and secondary (olive) spin magnitude distributions inferred when relaxing the IID assumption. We find both 
distributions to be remarkably similar, showing no signs that each component spin magnitude are distributed differently from each other.

\subsubsection{Spin Orientation}

The \textsc{Default} spin model assumes the spin orientation of both components are identical and independently distributed (IID) mixture models over an
aligned and an isotropic component. The aligned component is modeled with a truncated Gaussian distribution with mean at $\cos{\theta}=1$ and variance a free 
hyperparameter to be fit. This provides a simple 2-parameter model motivated by simple distributions expected from the two main formation scenario families, allowing 
for a straightforward interpretation of results. One possible limitation however, is that by construction this distribution is forced to peak at perfectly aligned spins, 
i.e. $\cos{\theta}=1$. While this may be a reasonable assumption, \citet{spinitasyoulike} recently extended the model space of parametric descriptions 
used to model the spin orientation distribution and found considerable evidence that the distribution peaks away from $\cos{\theta}=1$. Again, this provides a clear 
use-case where data-driven models can help in our understanding and possibly resolve disputes among differing parametric approaches.

\begin{figure}
    \begin{centering}
        \includegraphics[width=\linewidth]{figures/iid_spintilt.pdf}
        \caption{The spin orientation distribution inferred with the M-Spline model (red), with 16 knots spaced linearly from -1 to 1, and 
        assuming the components are IID. The solid line shows the population predictive distribution (PPD) and the shaded region, the 90\% credible interval. 
        We show the inferred PPD from the \textsc{Default} (blue) model from \citet{o3b_astro_dist}, the LVK's GWTC-3 population analyses.}
        \label{fig:iid_spintilt_dist}
    \end{centering}
    \script{plot_iid_spintilt.py}
\end{figure}

We repeat the tests done above on the spin magnitude distributions now with the spin orientation distributions. Figure \ref{fig:iid_spintilt_dist} 
shows the inferred spin magnitude distribution with the M-Spline model, compared with the \textsc{Default} model from \citet{o3b_astro_dist}. 
The M-Spline inferences have large uncertainties but start to show the same features as found and discussed in \citet{spinitasyoulike}. 
We find a distribution that instead of intrinsically peaking at $\cos{\theta}=1$, is found to peak at: $\cos{\theta}=$\result{$\CIPlusMinus{\macros[MSplineIIDCompSpins][peakCosTilt]}$}, at 
90\% credibility. We find less, but still considerable, support for misaligned spins (i.e. $\cos{\theta}<0$) as the \textsc{Default} model. Specifically we 
find that the fraction of misaligned systems is $f_{\cos{\theta}<0}=$\result{$\CIPlusMinus{\macros[MSplineIIDCompSpins][negFrac]}$}. This implies 
the presence of an isotropic component as expected by dynamical formation channels, albeit less than with the \textsc{Default} model. To quantify the 
amount of isotropy in the tilt distribution we calculate $\log_{10}Y$, where $Y$ is the ratio of nearly aligned tilts to nearly anti-aligned, 
introduced in \citet{spinitasyoulike} and defined as:

\begin{equation}
    Y \equiv \frac{\int_{0.9}^{1.0} d\cos{\theta} p(\cos{\theta})}{\int_{-1.0}^{-0.9} d\cos{\theta} p(\cos{\theta})}
\end{equation}

\noindent Now the log this quantity, $\log_{10}Y$, is 0 for tilt distribution that is purely isotropic, negative when anti-aligned values are favored, 
and positive when aligned tilts are favored. We find a $\log_{10}Y=$\result{$\CIPlusMinus{\macros[MSplineIIDCompSpins][log10gammaFrac]}$}, exhibiting a 
slight preference for aligned tilts.  

Now we model each of the components' orientation distributions with independent M-Spline models as done above, and show the inferred 
primary (orange), and secondary (olive) distributions in figure \ref{fig:ind_spintilt_dist}. The orientation distributions do not agree quite as well as 
the magnitude distributions did in figure \ref{fig:ind_spinmag_dist}, and the PPDs look to peak at different values, but the wide uncertainties still are 
consistent with both each other and the \textsc{Default} model's PPD. We find the two distributions to peak at: $\cos{\theta_1}=$\result{$\CIPlusMinus{\macros[MSplineIndependentCompSpins][peakCosTilt1]}$} 
and $\cos{\theta_2}=$\result{$\CIPlusMinus{\macros[MSplineIndependentCompSpins][peakCosTilt2]}$}, showing that the primary distribution peak is inferred further away 
from the assumed $\cos{\theta}=1$ with the \textsc{Default} model. There is also significant (albeit uncertain) evidence of spin misalignment in each distribution, finding 
the fraction of misaligned primary and secondary components as: $f_{\cos{\theta_1}<0}=$\result{$\CIPlusMinus{\macros[MSplineIndependentCompSpins][negFrac1]}$} and 
$f_{\cos{\theta_2}<0}=$\result{$\CIPlusMinus{\macros[MSplineIndependentCompSpins][negFrac2]}$}. We again calculate $\log_{10}Y$  
for each component distribution and find: $\log_{10}Y_1=$\result{$\CIPlusMinus{\macros[MSplineIndependentCompSpins][log10gammaFrac1]}$} and 
$\log_{10}Y_2=$\result{$\CIPlusMinus{\macros[MSplineIndependentCompSpins][log10gammaFrac2]}$}.

\begin{figure}
    \begin{centering}
        \includegraphics[width=\linewidth]{figures/ind_spintilt.pdf}
        \caption{The primary (orange) and secondary (olive) spin orientation distributions inferred with the M-Spline model, 
        with 16 knots spaced linearly from -1 to 1. The solid line shows the population predictive distribution (PPD) and the shaded region, the 90\% credible interval. 
        We show the inferred PPD from the \textsc{Default} (blue) model from \citet{o3b_astro_dist}, the LVK's GWTC-3 population analyses.}
        \label{fig:ind_spintilt_dist}
    \end{centering}
    \script{plot_ind_spintilt.py}
\end{figure}

\subsection{The Effective Spin Dimension}

\begin{figure} 
    \begin{centering}
        \includegraphics[width=\linewidth]{figures/chi_eff.pdf}
        \caption{The effective spin distribution inferred with the M-Spline model (red), with 24 knots spaced linearly from -1 to 1. 
        The solid line shows the population predictive distribution (PPD) and the shaded region, the 90\% credible interval. We show the inferred $\chi_\mathrm{eff}$ 
        PPD from the IID component spin M-Spline model (green), the independent component spin M-Spline model (purple), 
        and both the \textsc{Default} (blue) model and the Gaussian $\chi_\mathrm{eff}$ (olive) model from \citet{o3b_astro_dist}, the LVK's GWTC-3 population analyses.}
        \label{fig:chieff_dist}
    \end{centering}
    \script{plot_chieff.py}
\end{figure}

While the component spin magnitude and tilts are more easily interpretable in context of the astrophysics, they are parameters that 
are not measured well during individual event parameter estimation. The best measured spin parameter, that enters the waveform at XXPN, is 
called the effective spin defined as: $\chi_\mathrm{eff} = \frac{a_1\cos{\theta_1} + qa_2\cos{\theta_2}}{1+q}$. As an alternative check 
on the spin distributions we also try modeling an M-Spline distribution of effective spins instead of component spin magnitude and tilts 
as done above. Figure \ref{fig:chieff_dist} shows the inferred effective spin distribution with the M-Spline model (red) along with the 
inferred effective spin distribution transformed from the PPDs of the component spin models shown previously. We find considerable agreement 
among the component spin distributions, highlighting that different component spin distributions can imply very similar effective spin 
distributions which is the better constrained parameter. When modeling the effective spin with the M-Spline model we see very similar 
shape, but with a sharper peak centered at $\chi_\mathrm{eff}=$\result{$\CIPlusMinus{\macros[ChiEffective][chieff][PeakChiEff]}$}, compared to 
$\chi_\mathrm{eff}=$\result{$\CIPlusMinus{\macros[ChiEffective][default][PeakChiEff]}$} with the \textsc{Default} model and 
$\chi_\mathrm{eff}=$\result{$\CIPlusMinus{\macros[ChiEffective][gaussian][PeakChiEff]}$} with the Gaussian $\chi_\mathrm{eff}$ model. 
As for the misaligned spin question, we calculate the fraction of systems with effective spins that are misaligned (i.e. $\chi_\mathrm{eff}<0$) and 
find for the M-Spline model $f_{\chi_\mathrm{eff}<0}=$\result{$\CIPlusMinus{\macros[ChiEffective][chieff][FracBelow0]}$}, compared to 
$f_{\chi_\mathrm{eff}<0}=$\result{$\CIPlusMinus{\macros[ChiEffective][default][FracBelow0]}$} with the \textsc{Default} model and 
$f_{\chi_\mathrm{eff}<0}=$\result{$\CIPlusMinus{\macros[ChiEffective][gaussian][FracBelow0]}$} with the Gaussian $\chi_\mathrm{eff}$ model.
