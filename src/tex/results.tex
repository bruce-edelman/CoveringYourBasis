\section{Population Inference of GWTC-3 BBHs} \label{sec:results}

\subsection{Mass Distribution} \label{sec:mass_dist}

We apply our newly constructed models to the BBH analysis that was done in the astrophysical population companion paper to GWTC-3. 
We find consistent features in the mass disribution with what was inferred with the \textsc{PowerlawPeak} and \textsc{PowerlawSpline} 
mass models. In particular our MSpline model finds a peak in merger rate density at both 10 and 35 soloar masses, agreeing with those reported
in \bruce{CITE o3b paper}. We also find a significantly less declining mass distribution that has considerably larger support for higher mass 
mergers when compared to the other power-law based models. Another interesting consequence of this is that we thus infer a much less steeply
evolving merger rate with redshift. It is known that there is a strong correlation between the primary mass power law index, $alpha$, and the 
redshift power-law index, $\lambda_z$. When one infers a steeper mass distribution -- the inferred rate must drastically increase 
with resdshift to explain the few higher mass events in our catalog, but when one infers a flatter distribution like the MSPline inferred
distribition in plot \ref{fig:mass_distribution}, the higher mass mergers are consistent with the local merger rate denstiy as a function of mass, 
and thus the model does not need to have as steep of power-law in redshift. With this more flexible mass population model, we now 
find that the merger rate is fully consistent with being unifom with co-moving volume, rather than the confidently evolving merger rate 
conclusion from \bruce{cite O3b}. 

\begin{figure*}[ht!]
    \script{mass_distribution_plot.py}
    \begin{centering}
        \includegraphics[width=\textwidth]{figures/mass_distribution_plot.pdf}
        \caption{Plot showing the primary mass (left) and mass ratio (right) distributions inferred with the 
        \textsc{MSpline} model (black) with 40 knots for $m_1$, and 15 knots for $q$. We show the results for the \textsc{PowerlawPeak} (blue) and 
        \textsc{PowerlawSpline} (orange) models fromthe LVK's GWTC-3 population analyses. Solid lines show the median of the posterior while the shaded
        bands show the 90\% credible intervals.}
        \label{fig:mass_distribution}
    \end{centering}
\end{figure*}


In addition to the primary mass distirbution there is some mild evidence for a mass ratio distibution that is not well described by a power law, 
which is how the mass ratio distribution is commonly modeled. In particular our smoothed MSpline model finds a decrease in the merger rate at $q\sim1$, 
while agreeing well with the power-law on the decline in rate from $q \in [0.4, 0.8]$, and a much larger tail extending to more asymmetric
mass ratios. 

\subsection{Spin Distributions} \label{sec:spin_dist}

The spins distribution of merging combact binaries provides a very useful insight into the formation history of the binary. In particular the two main subcategories, 
the isolated and dynamical formation channels, would produce distinct signatures in the spin distribution when looking at a population of observations. Isolated formation
is where the only interactions with the binary are between each other. As the binary evolves and mass transfer happens between the two components, we expect the orientation
of each components spin vector to be preferentially aligned with the orbital angular mommentumof the system. Some misalignment is epected to occur due to the kicks from 
each star exploding in supernovae leaving behind the compact objects. Conversely, the dynamical channel     

\begin{figure*}[ht!]
    \script{component_spin_distribution_plot.py}
    \begin{centering}
        \includegraphics[width=\textwidth]{figures/component_spin_distribution_plot.pdf}
        \caption{Plot showing the inferred component spin magnitude (top) and spin tilt (bottom) distributions inferred with the 9-knot \textsc{MSpline} models. 
        There are two independent \textsc{MSpline} (black) models for each primary/secondary spin magnitude and spin tilt. We compare by showing the results for the
        \textsc{Default} (iid) spin model used with the \textsc{PowerlawPeak} mass model (blue) and the \textsc{PowerlawSpline} mass model (orange).}
        \label{fig:component_spin_distribution}
    \end{centering}
\end{figure*}

\begin{figure*}
    \script{iid_component_spin_distribution_plot.py}
    \begin{centering}
        \includegraphics[width=\textwidth]{figures/iid_component_spin_distribution_plot.pdf}
        \caption{Plot showing the inferred component spin magnitude (left) and spin tilt (right) distributions inferred with the 9-knot \textsc{MSpline} models. 
        Each components spin magnitude and tilt are assumed to be IID. We compare by showing the results for the
        \textsc{Default} (iid) spin model used with the \textsc{PowerlawPeak} mass model (blue) and the \textsc{PowerlawSpline} mass model (orange).}
        \label{fig:iid_component_spin_distribution}
    \end{centering}
\end{figure*}