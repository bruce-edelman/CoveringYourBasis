\section{Binary Black Hole Population Inference with GWTC-3} \label{sec:results}

\begin{figure*}[ht!]
    \begin{centering}
        \includegraphics[width=\linewidth]{figures/mass_distribution_plot.pdf}
        \caption{The primary mass distribution inferred with the \textsc{MSpline} model (red), with 50 knots spaced lineary in $\log m_1$, from the 
        minimum observed value to 100\msun. The solid line shows the population predictive distribution (PPD) and the shaded region, 
        the 90\% credible interval. We show the inferred PPD from the \textsc{PowerlawSpline} (blue) model from \citet{o3b_astro_dist}, the LVK's GWTC-3 population analyses.}
        \label{fig:mass_distribution}
    \end{centering}
    \script{mass_distribution_plot.py}
\end{figure*}

We use hierarchical Bayesian inference (see Appendix \ref{sec:hierarchical_inference}) to simultaneosly infer the astrophysical mass, spin and redshift distributions of 
a binary black holes (BBHs) given a catalog of gravitational wave observations. We parameterize the binaries' masses with the primary (defined as the more massive component) mass 
and the mass ratio -- defined as $q=\frac{m_2}{m_1}$, bounded in the range from 0 to 1. We model 4 of the 6 total 
spin degrees of freedom of a binary merger with each component spin magnitude $a_1$ and $a_2$, along with the cosine of the tilt angle of each component, 
$\cos{\theta_1}$ and $\cos{\theta_2}$. The tilt angle is defined as the angle between each components' spin vector and the binaries orbital angular mommentum vector. 
We additionally will fit a population model on the redshift or luminosity distance distribution of BBHs. We assume a $\Lambda\mathrm{CDM}$ cosmology using the 
parameters from the Planck 2015 results \cite{Planck2015}, which defines an analytical mapping between each event's inferred
luminosity distance and it's redshift. While we choose to use MSpline distibutions to model the primary mass and spin distributions, we 
model the mass ratio distribution with a power law, i.e. $p(q | m_1, m_\mathrm{min}, \beta) \propto q^{\beta} \Theta(qm_1 - m_\mathrm{min}) \Theta(m_1 - qm_1)$. 
Where $\beta$ is the power law index and $\Theta$ is the Heaviside step function which ensures $m_2$ is within the allowed range, [$m_\mathrm{min}$, $m_1$] \cite{Talbot_2018,o1o2_pop,o3a_pop}.
We model the evolution of the merger rate with redshift with the \textsc{PowerlawRedshift} model \cite{Fishbach_2018redshift}, which is desribed by 
$p(z|\lambda_z)\propto \frac{dV_c}{dz}(1+z)^{\lambda_z-1}$. We describe our specific prior choices on the hyper-parameters and each model 
in more detail in Appendix \ref{sec:modelpriors}. We use the same cut on the recent GWTC-3 catalog as was done in the LVK's accompanying population analysis, 
keeping all 70 possibly BBH mergers with false alarm rates less than 1 per year \cite{GWTC3,o3b_astro_dist}. We choose to emit the currently not well understood event, 
GW190814 \cite{190814disc}, as it was concluded to be an outlier of the rest of the BBH population in both GWTC-2 and GWTC-3 \cite{o3a_pop,o3b_astro_dist,Essick_2022}. 
This leaves us with a catalog of 69 BBH mergers observerd oveer a period of about 2 years, from which we want to infer its population properties. 
We start by looking towards the mass distribution.

\subsection{Binary Black Hole Masses} \label{sec:mass_dist}

We apply our newly constructed models to the BBH analysis that was done in the astrophysical population companion paper to GWTC-3. 
Figure \ref{fig:mass_distribution} shows how the MSpline model (red) found consistent features in the mass disribution with 
what was inferred with the \textsc{PowerlawPeak} and \textsc{PowerlawSpline} mass models \cite{Talbot_2018,o3a_pop,Edelman_2022ApJ,o3b_astro_dist}. 
In particular our MSpline model finds a peak in merger rate density at both 10 and 35 soloar masses, agreeing with those reported using the same dataset in \citet{o3b_astro_dist}. 
We also find a significantly less declining mass distribution that has considerably larger support for higher mass mergers when 
compared to the other power-law based models, finding that 75\% percent of primary masses below \result{$\CIPlusMinus{\macros[MassDistribution][MSpline][m_75percentile]}$}, 
at 90\% credibility, compared to \result{$\CIPlusMinus{\macros[MassDistribution][PLSpline][m_75percentile]}$}, found with the similar \textsc{PowerlawSpline} model. 
An interesting consequence of this is that we do not find a confidently evolving merger rate with redshift, as inferred with power-law based mass models in \citet{o3b_astro_dist}. 
There is a strong correlation between the primary mass power law index, $\alpha$, and the redshift power-law index, $\lambda_z$. When one infers a steeper mass distribution -- 
the inferred rate must drastically increase with resdshift to explain the highest mass events in the catalog. However, when one infers a 
flatter distribution like, as the MSPline did in figure \ref{fig:mass_distribution}, the higher mass mergers are consistent with 
the local merger rate denstiy as a function of mass given the selection effects. With our more flexible mass model, we find that the merger rate 
is consistent with being unifom with co-moving volume, with $\lambda_z = $\result{$\CIPlusMinus{\macros[MSplineIIDCompSpins][lamb]}$}, compared with 
\citet{o3b_astro_dist}'s reported value -- $\lambda_z = $\result{$\CIPlusMinus{\macros[PLPeak][lamb]}$} with the \textsc{PowerlawPeak} mass model.

We find a similar mass ratio distribution to what was found with the found with the \textsc{PowerlawPeak} model in \citet{o3b_astro_dist}. 
With the MSpline model on primary mass the mass ratio power law index is $\beta = $\result{$\CIPlusMinus{\macros[MSplineIIDCompSpins][beta]}$}, 
compared to $\beta = $\result{$\CIPlusMinus{\macros[PLPeak][beta]}$} for \textsc{PowerlawPeak}. We find that BBH mergers favor more equal mass mergers
as would be expected of both main formation channels due to binary mass transfer in isolated formation and three-body interactions that preferentially eject
the lightest of the three compact objects, in dynamical formation \bruce{cite these}. We also find no evidence for a sharp fall off in merger rate following 
the pileup at $\sim35\,\msun$, which is expected if such a pileup was due to pulsational pair instability supernovae (PPISNe). The lack of any high mass truncation 
along with the peak at 35\msun, significantly lower than expected, may pose challenges for conventional stellar evolution theory and could be a sign of 
the significant presence of sub-populations that may avoid PISN as the binary forms. 

\subsection{Binary Black Hole Spins} \label{sec:spin_dist}

\begin{figure}
    \begin{centering}    
        \includegraphics[width=\linewidth]{figures/iid_spinmag.pdf}
        \caption{The spin magnitude distribution inferred with the \textsc{MSpline} model (red), with 16 knots spaced linearly from 0 to 1, and 
        assuming the components are IID. The solid line shows the population predictive distribution (PPD) and the shaded region, the 90\% credible interval. 
        We show the inferred PPD from the \textsc{Default} (blue) model from \citet{o3b_astro_dist}, the LVK's GWTC-3 population analyses.}
        \label{fig:iid_spinmag_dist}
    \end{centering}
    \script{plot_iid_spinmag.py}
\end{figure}

\subsubsection{Spin Magnitude}

The \textsc{Default} spin model models the spin magnitude of both components as identical and indendendently distributed (IID) non-singular Beta distributions. The
Beta distribution provides a simple 2-parameter model that can produce a wide range of functional forms on the unit interval. However, the constraint that keeps 
the Beta distribution non-singluar (i.e. $\alpha>1$ and $\beta>1$) enforces a spin magnitude that always has $p(a_i=0) = 0$. While in reality, it is expected that 
all physical bodies must have at least some non-zero angular mommentum, studies have proposed the existence of a distinct sub-population of non-spinning or 
negliglby spinning black holes that can elude discovery with such a model. 

\begin{figure} 
    \begin{centering}
        \includegraphics[width=\linewidth]{figures/ind_spinmag.pdf}
        \caption{The primary (orange) and secondary (olive) spin magnitude distributions inferred with the \textsc{MSpline} model, 
        with 16 knots spaced linearly from 0 to 1. The solid line shows the population predictive distribution (PPD) and the shaded region, the 90\% credible interval. 
        We show the inferred PPD from the \textsc{Default} (blue) model from \citet{o3b_astro_dist}, the LVK's GWTC-3 population analyses.}
        \label{fig:ind_spinmag_dist}
    \end{centering}
    \script{plot_ind_spinmag.py}
\end{figure}

We model the spin magnitude distibutions as IID MSpline distibutions as a flexible check
to the assumptions of the \textsc{Default} model. Figure \ref{fig:iid_spinmag_dist} shows the inferred spin magnitude distribution with the \textsc{MSpline} model, 
compared with the \textsc{Default} model from \citet{o3b_astro_dist}. The MSpline model finds broad agreement with the beta distribution, both peaking near ~0.2, with 90\% 
of BBH spins below \result{$\CIPlusMinus{\macros[MSplineIIDCompSpins][a_90percentile]}$} at 90\% credibility. The regions of parameter space where they don't agree as well are the bounds. 
The MSpline model has no enforcment for vanishing support at the extremal values like the Beta distribution, 
allowing it to probe the zero-spin question. We find broad support, with large variance, for non-zero probabilities at $a_i=0$, but cannot confidently determine the presence of 
a significant non-spinning subpopulation, corroborating the conclusions of \citet{BuildBetterSpinModels,Callister_NoEvidence,GWTC3MonashSpin}. 
We now repeat the same analysis as above but with independent MSpline distibutions for each spin magnitude component. In figure \ref{fig:ind_spinmag_dist} 
we show the inferred primary (orange), and secondary (olive) spin magnitude distributions inferred when relaxing the IID assumption. We find both 
distributions to be remarkably similar, showing no signs that each component spin magnitude are distributed differently from each other.

\subsubsection{Spin Orientation}

The \textsc{Default} spin model models the spin orientation of both components as identical and indendendently distributed (IID) mixture models over an
aligned and an isotropic component. The aligned component is modeled with a truncated Gaussian distribution with mean at $\cos{\theta}=1$ and variance a free 
hyper-parameter to be fit. This provides a simple 2-parameter model motivated by simple distributions expected from the two main formation scenario families, allowing 
for a straightforward interpretation of results. One possible limitation however, is that by construction this distribution is forced to peak at perfecty aligned spins, 
i.e. $\cos{\theta}=1$. While this may be a reasonable assumption recently \citet{spinitasyoulike} recently extended the model space of parameteric descriptions 
used to model the spin orientation distribution and found considerably evidence that the distibution peaks away from $\cos{\theta}=1$. Again, this provides a clear 
use-case where data-driven models can help in our understanding and possibly resolve disputes among differing parametric approaches.

\begin{figure}
    \begin{centering}
        \includegraphics[width=\linewidth]{figures/iid_spintilt.pdf}
        \caption{The spin orientation distribution inferred with the \textsc{MSpline} model (red), with 16 knots spaced linearly from -1 to 1, and 
        assuming the components are IID. The solid line shows the population predictive distribution (PPD) and the shaded region, the 90\% credible interval. 
        We show the inferred PPD from the \textsc{Default} (blue) model from \citet{o3b_astro_dist}, the LVK's GWTC-3 population analyses.}
        \label{fig:iid_spintilt_dist}
    \end{centering}
    \script{plot_iid_spintilt.py}
\end{figure}

We repeat the tests done above on the the spin magnitude distibutions now with the spin orientation distributions. Figure \ref{fig:iid_spintilt_dist} 
shows the inferred spin magnitude distribution with the \textsc{MSpline} model, compared with the \textsc{Default} model from \citet{o3b_astro_dist}. 
The MSpline inferences have large uncertainties but start to show the same features as found and discussed in \citet{spinitasyoulike}. 
We find a distribution that instead of intrinsicly peaking at $\cos{\theta}=1$, is found to peak at: $\cos{\theta}=$\result{$\CIPlusMinus{\macros[MSplineIIDCompSpins][peakCosTilt]}$}, at 
90\% credibility. We find less, but still considerable, support for misaligned spins (i.e. $\cos{\theta}<0$) as the \textsc{Default} model. Specifically we 
find that the fraction of misaligned systems is $f_{<0}=$\result{$\CIPlusMinus{\macros[MSplineIIDCompSpins][negFrac]}$}. This implies 
the presence of an isotropic component as expected by dynamical formation channels, albeit less than with the \textsc{Default} model. To quantify the 
amount of isotropy in the tilt distibution we calculate Y, the ratio defined in Eq. 4 of \citet{spinitasyoulike} which is 1 when the distribution is 
purely isotropic, less than one when negative (anti-aligned) values are favored, and greater than one when positive (aligned) values are favored.
We find a $\log_{10}Y=$\result{$\CIPlusMinus{\macros[MSplineIIDCompSpins][log10gammaFrac]}$}. 

Now we model each of the components' orientation distributions with independent \textsc{MSpline} models as done above, and show the inferred 
primary (orange), and secondary (olive) distributions in figure \ref{fig:ind_spintilt_dist}. The orientation distibutions do not agree quite as well as 
the magnitude distibutions did in figure \ref{fig:ind_spinmag_dist}, and the PPDs look to peak at different values, but the wide uncertainties still are 
consistent with both each other and the \textsc{Default} model's PPD. We find the two distributions to peak at: $\cos{\theta_1}=$\result{$\CIPlusMinus{\macros[MSplineIndependentCompSpins][peakCosTilt1]}$} 
and $\cos{\theta_2}=$\result{$\CIPlusMinus{\macros[MSplineIndependentCompSpins][peakCosTilt2]}$}, showing that the primary distribution peak is inferred further away 
from the assumed $\cos{\theta}=1$ with the \textsc{Default} model. There is also significant (albeit uncertain) evidence of spin mis-alignments in each distribution, finding 
the fraction of mis-aligned primary and secondary components as: $f_{1,<0}=$\result{$\CIPlusMinus{\macros[MSplineIndependentCompSpins][negFrac1]}$} andand $f_{2,<0}=$\result{$\CIPlusMinus{\macros[MSplineIndependentCompSpins][negFrac2]}$}.
We again calculate the isotropy ratio, Y, for each and find: $\log_{10}Y_1=$\result{$\CIPlusMinus{\macros[MSplineIndependentCompSpins][log10gammaFrac1]}$} and 
$\log_{10}Y_2=$\result{$\CIPlusMinus{\macros[MSplineIndependentCompSpins][log10gammaFrac2]}$}.

\begin{figure}
    \begin{centering}
        \includegraphics[width=\linewidth]{figures/ind_spintilt.pdf}
        \caption{The primary (orange) and secondary (olive) spin orientation distributions inferred with the \textsc{MSpline} model, 
        with 16 knots spaced linearly from -1 to 1. The solid line shows the population predictive distribution (PPD) and the shaded region, the 90\% credible interval. 
        We show the inferred PPD from the \textsc{Default} (blue) model from \citet{o3b_astro_dist}, the LVK's GWTC-3 population analyses.}
        \label{fig:ind_spintilt_dist}
    \end{centering}
    \script{plot_ind_spintilt.py}
\end{figure}

\subsection{The Effective Spin Dimension}

\begin{figure} 
    \begin{centering}
        \includegraphics[width=\linewidth]{figures/chi_eff.pdf}
        \caption{The effective spin distribution inferred with the \textsc{MSpline} model (red), with 24 knots spaced linearly from -1 to 1. 
        The solid line shows the population predictive distribution (PPD) and the shaded region, the 90\% credible interval. We show the inferred $\chi_\mathrm{eff}$ 
        PPD from the IID component spin \textsc{MSpline} model (green), the independent component spin \textsc{MSpline} model (purple), 
        and the \textsc{Default} (blue) model from \citet{o3b_astro_dist}, the LVK's GWTC-3 population analyses.}
        \label{fig:chieff_dist}
    \end{centering}
    \script{plot_chieff.py}
\end{figure}

While the component spin magnitude and tilts are more easily interpretable in context of the astrophysics, they are parameters that 
are not measured well during individual event parameter estimation. The best measured spin parameter, that enters the waveform at XXPN, is 
called the effective spin defined as: $\chi_\mathrm{eff} = \frac{a_1\cos{\theta_1} + qa_2\cos{\theta_2}}{1+q}$. As an alternative check 
on the spin distributions we also try modeling an MSpline distribution of effective spins instead of component spin magnitude and tilts 
as done above. Figure \ref{fig:chieff_dist} shows the inferred effective spin distribution with the MSpline model (red) along with the 
inferred effective spin distribution transformed from the PPDs of the component spin models shown previously. We find considerable agreement 
among the component spin distributions, highlighting that wildy different component spin distributions can imply very similar effective spin 
distributions which is the better constrained parameter. When modeling the effective spin with the \textsc{MSpline} model we see very similar 
shape, but with a slightly sharper peak centered at $\chi_\mathrm{eff}=$\result{$\CIPlusMinus{\macros[ChiEffective][chieff][PeakChiEff]}$}, compared to 
$\chi_\mathrm{eff}=$\result{$\CIPlusMinus{\macros[ChiEffective][default][PeakChiEff]}$} with the \textsc{Default} model. As for the misaligned spin question, 
we can investigate the fraction of systems with effective spins that are mis-aligned (i.e. $\chi_\mathrm{eff}<0$) and find for the \textsc{MSpline} 
model $f_{\chi_\mathrm{eff}<0}=$\result{$\CIPlusMinus{\macros[ChiEffective][chieff][FracBelow0]}$}, compared to $f_{\chi_\mathrm{eff}<0}=$\result{$\CIPlusMinus{\macros[ChiEffective][default][FracBelow0]}$} 
with the \textsc{Default} model.
